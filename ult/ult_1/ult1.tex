%%%%%%%%%%%%%%%%%%%%%%%%%%%%%%%%%%%%%%%%%%%%%%%%%%%%%%%%%%%%%%%%%%%%%%%%%%%%%%%%
% Using Linux Tools
%
% Author: FOSSEE 
% Copyright (c) 2009, FOSSEE, IIT Bombay
%%%%%%%%%%%%%%%%%%%%%%%%%%%%%%%%%%%%%%%%%%%%%%%%%%%%%%%%%%%%%%%%%%%%%%%%%%%%%%%%

\documentclass[12pt,compress]{beamer}

\mode<presentation>
{
  \usetheme{Warsaw}
  \useoutertheme{infolines}
  \setbeamercovered{transparent}
}

\usepackage[english]{babel}
\usepackage[latin1]{inputenc}
%\usepackage{times}
\usepackage[T1]{fontenc}

% Taken from Fernando's slides.
\usepackage{ae,aecompl}
\usepackage{mathpazo,courier,euler}
\usepackage[scaled=.95]{helvet}

\definecolor{darkgreen}{rgb}{0,0.5,0}

\usepackage{listings}
\lstset{language=sh,
    basicstyle=\ttfamily\bfseries,
    commentstyle=\color{red}\itshape,
  stringstyle=\color{darkgreen},
  showstringspaces=false,
  keywordstyle=\color{blue}\bfseries}

%%%%%%%%%%%%%%%%%%%%%%%%%%%%%%%%%%%%%%%%%%%%%%%%%%%%%%%%%%%%%%%%%%%%%%
% DOCUMENT STARTS
\begin{document}

\begin{frame}

\begin{center}
\vspace{12pt}
\textcolor{blue}{\huge Using Linux Tools\\Part I}
\end{center}
\vspace{18pt}
\begin{center}
\vspace{10pt}
\includegraphics[scale=0.95]{../images/fossee-logo.png}\\
\vspace{5pt}
\scriptsize Developed by FOSSEE Team, IIT-Bombay. \\ 
\scriptsize Funded by National Mission on Education through ICT\\
\scriptsize  MHRD,Govt. of India\\
\includegraphics[scale=0.30]{../images/iitb-logo.png}\\
\end{center}
\end{frame}

\begin{frame}
\frametitle{Objectives}
\label{sec-2}

At the end of this tutorial, you will be able to,
\begin{itemize}
\item Know what is linux.
\item  Understand the need for linux in today's world.
\item Move around in directories and files.
\item Use basic commands of Linux.
\end{itemize}
\end{frame}

\begin{frame}[fragile]
  \begin{block}{What is the Linux OS?}
    \begin{itemize}
    \item Free Open Source Operating System
      \begin{description}
        \item[Free] 
          Free as in Free Speech
        \item[Open-Source]
          Permit modifications and redistribution of source code
      \end{description}
    \item Also called GNU/Linux 
    \item Unix-inspired
    \item Runs on a variety of hardware
    \item Linux Kernel + Application software
    \end{itemize}
  \end{block}
\end{frame}

\begin{frame}[fragile]
  \frametitle{Why Linux?}
    \begin{itemize}
    \item Free 
    \item Secure \& versatile
    \end{itemize}

    \begin{block}{Why Linux for Scientific Computing?}
      \begin{itemize}
        \item Free as in Free Speech
        \item Can run for \emph{ever}
        \item Libraries
        \item Parallel Computing
      \end{itemize}
    \end{block}
\end{frame}

\begin{frame}[fragile]
  \frametitle{Logging in}
  \begin{itemize}
  \item GNU/Linux does have a GUI
  \item Command Line for this module
  \item Hit \texttt{Ctrl + Alt + F1}
  \item \texttt{logout} command logs you out
  \end{itemize}
\end{frame}

\begin{frame}[fragile]
  \frametitle{Creating folders}
  \begin{itemize}
  \item Special characters need to be escaped OR quoted
  \end{itemize}
  \begin{lstlisting}
    $ mkdir software\ engineering
    $ mkdir "software engg"
  \end{lstlisting} 
  \begin{itemize}
  \item Generally, use hyphens or underscores instead of spaces in names
  \end{itemize}
\end{frame}

\begin{frame}[fragile]
  \frametitle{Using additional options}

  \begin{itemize}
  \item \texttt{-h} or \texttt{--help} gives summary of command usage
  \end{itemize}
  \begin{lstlisting}
    $ ls --help
  \end{lstlisting} % $
\end{frame}

\begin{frame}
  \frametitle{Exercise 1}
  \begin{itemize}
  \item Which option should be used with ``ls'' command to list all the directories,
        sub-directories and files contained in it? 
  \end{itemize}
        Hint: Use ``man'' or ``--help''
\end{frame}

\begin{frame}
\frametitle{Summary}
\label{sec-8}

  In this tutorial, we have learnt to,


\begin{itemize}
\item Understand the basic structure of linux and it's need.
\item See the current directory in which we are working, using the command ``pwd''.
\item List a directory's contents by using the command ``ls''.
\item Change file timestamps, using the command ``touch''.
 \item Use commands like ``mkdir'' and ``rmdir'' to make and remove directories 
      respectively.
\item Use commands such as ``man'' and ``whatis'' to get a description of 
      what a particular command does.
\item Search the manual page names and descriptions, using the `` apropos'' command.
\end{itemize}
\end{frame}
\begin{frame}[fragile]
\frametitle{Evaluation}
\label{sec-9}


\begin{enumerate}
\item Which is the default directory after logging into the terminal?
\vspace{8pt}
\item How to view file attributes with ``ls'' command? 
\end{enumerate}
\end{frame}
\begin{frame}
\frametitle{Solutions}
\label{sec-10}


\begin{enumerate}
\item /home/user
\vspace{15pt}
\item ls -l <filename>
\end{enumerate}
\end{frame}
\begin{frame}

  \begin{block}{}
  \begin{center}
  \textcolor{blue}{\Large THANK YOU!} 
  \end{center}
  \end{block}
\begin{block}{}
  \begin{center}
    For more Information, visit our website\\
    \url{http://fossee.in/}
  \end{center}  
  \end{block}
\end{frame}

\end{document}

