%%%%%%%%%%%%%%%%%%%%%%%%%%%%%%%%%%%%%%%%%%%%%%%%%%%%%%%%%%%%%%%%%%%%%%%%%%%%%%%%
% Using Linux Tools
%
% Author: FOSSEE 
% Copyright (c) 2009, FOSSEE, IIT Bombay
%%%%%%%%%%%%%%%%%%%%%%%%%%%%%%%%%%%%%%%%%%%%%%%%%%%%%%%%%%%%%%%%%%%%%%%%%%%%%%%%

\documentclass[12pt,compress]{beamer}

\mode<presentation>
{
  \usetheme{Warsaw}
  \useoutertheme{infolines}
  \setbeamercovered{transparent}
}

\usepackage[english]{babel}
\usepackage[latin1]{inputenc}
%\usepackage{times}
\usepackage[T1]{fontenc}

% Taken from Fernando's slides.
\usepackage{ae,aecompl}
\usepackage{mathpazo,courier,euler}
\usepackage[scaled=.95]{helvet}

\definecolor{darkgreen}{rgb}{0,0.5,0}

\usepackage{listings}
\lstset{language=sh,
    basicstyle=\ttfamily\bfseries,
    commentstyle=\color{red}\itshape,
  stringstyle=\color{darkgreen},
  showstringspaces=false,
  keywordstyle=\color{blue}\bfseries}

%%%%%%%%%%%%%%%%%%%%%%%%%%%%%%%%%%%%%%%%%%%%%%%%%%%%%%%%%%%%%%%%%%%%%%
% DOCUMENT STARTS
\begin{document}

\begin{frame}

\begin{center}
\vspace{12pt}
\textcolor{blue}{\huge Using Linux Tools\\Part VII}
\end{center}
\vspace{18pt}
\begin{center}
\vspace{10pt}
\includegraphics[scale=0.95]{../images/fossee-logo.png}\\
\vspace{5pt}
\scriptsize Developed by FOSSEE Team, IIT-Bombay. \\ 
\scriptsize Funded by National Mission on Education through ICT\\
\scriptsize  MHRD,Govt. of India\\
\includegraphics[scale=0.30]{../images/iitb-logo.png}\\
\end{center}
\end{frame}
\begin{frame}
\frametitle{Objectives}
\label{sec-2}

At the end of this tutorial, you will be able to,
\begin{itemize}
\item Prepare scripts using 'Control Operators'.
\item Understand what 'Environment Variables' are.
\end{itemize}
\end{frame}

\begin{frame}
\frametitle{Pre-requisite}
\label{sec-3}

Spoken tutorial on -
\begin{itemize}
\item Using Linux tools -- Part I
\item Using Linux tools -- Part II
\item Using Linux tools -- Part III
\item Using Linux tools -- Part IV
\item Using Linux tools -- Part V
\item Using Linux tools -- Part VI
\end{itemize}
\end{frame}

\begin{frame}[fragile]
  \frametitle{\texttt{if}}
  \begin{itemize}
  \item Print message if directory exists in \texttt{pwd}
  \end{itemize}
  \begin{lstlisting}
    #!/bin/bash
    if test -d $1
    then
    echo "Yes, the directory" \
    $1 "is present"
    fi
  \end{lstlisting} % $
\end{frame}

\begin{frame}[fragile]
  \frametitle{\texttt{[ ]} - alias for \texttt{test}}
  \begin{itemize}
  \item Square brackets (\texttt{[]}) can be used instead of
    \texttt{test}
  \item 
  \end{itemize}
  \begin{lstlisting}
    #!/bin/bash
    if [ $1 -lt 0 ]
    then
    echo "number is negative"
    else
    echo "number is non-negative"
    fi
  \end{lstlisting} % $
  \begin{itemize}
  \item \alert{spacing is important, when using the square brackets}
  \end{itemize}
\end{frame}

\begin{frame}[fragile]
  \frametitle{\texttt{for}}
  \begin{block}{Problem}
    Given a set of \texttt{.mp3} files, that have names beginning with
    numbers followed by their names --- \texttt{08 - Society.mp3} ---
    rename the files to have just the names. Also replace any spaces
    in the name with hyphens. 
  \end{block}
  \begin{itemize}
  \item Loop over the list of files
  \item Process the names, to get new names
  \item Rename the files
  \end{itemize}
\end{frame}

\begin{frame}[fragile]
  \frametitle{Environment Variables}
  \begin{itemize}
  \item Pass information from shell to programs running in it
  \item Behavior of programs can change based on values of variables
  \item Environment variables vs. Shell variables
  \item Shell variables -- only current instance of the shell
  \item Environment variables -- valid for the whole session
  \item Convention -- environment variables are UPPER CASE
  \end{itemize}
\end{frame}

\begin{frame}
\frametitle{Summary}
\label{sec-8}

  In this tutorial, we have learnt to,


\begin{itemize}
\item Prepare scripts using control structures like ``if'', ``if-else'',
      ``for'' and ``while''.
\item Use 'environment variables'.
\item Export a variable to the environment of all the processes, using 
      the ``export'' command.
\end{itemize}
\end{frame}
\begin{frame}[fragile]
\frametitle{Evaluation}
\label{sec-9}


\begin{enumerate}
\item Print the text ``dog man'' in such a way that the prompt 
    continues after the text.
\vspace{8pt}
\item How can you add a new path variable ``/data/myscripts'' to \$PATH variable ?
\end{enumerate}
\end{frame}
\begin{frame}
\frametitle{Solutions}
\label{sec-10}


\begin{enumerate}
\item \$ echo -n dog man
\vspace{15pt}
\item \$ export PATH=\$PATH://data/myscripts
\end{enumerate}
\end{frame}
\begin{frame}

  \begin{block}{}
  \begin{center}
  \textcolor{blue}{\Large THANK YOU!} 
  \end{center}
  \end{block}
\begin{block}{}
  \begin{center}
    For more Information, visit our website\\
    \url{http://fossee.in/}
  \end{center}  
  \end{block}
\end{frame}

\end{document}


