%%%%%%%%%%%%%%%%%%%%%%%%%%%%%%%%%%%%%%%%%%%%%%%%%%%%%%%%%%%%%%%%%%%%%%%%%%%%%%%%
% Using Linux Tools
%
% Author: FOSSEE 
% Copyright (c) 2009, FOSSEE, IIT Bombay
%%%%%%%%%%%%%%%%%%%%%%%%%%%%%%%%%%%%%%%%%%%%%%%%%%%%%%%%%%%%%%%%%%%%%%%%%%%%%%%%

\documentclass[12pt,compress]{beamer}

\mode<presentation>
{
  \usetheme{Warsaw}
  \useoutertheme{infolines}
  \setbeamercovered{transparent}
}

\usepackage[english]{babel}
\usepackage[latin1]{inputenc}
%\usepackage{times}
\usepackage[T1]{fontenc}

% Taken from Fernando's slides.
\usepackage{ae,aecompl}
\usepackage{mathpazo,courier,euler}
\usepackage[scaled=.95]{helvet}

\definecolor{darkgreen}{rgb}{0,0.5,0}

\usepackage{listings}
\lstset{language=sh,
    basicstyle=\ttfamily\bfseries,
    commentstyle=\color{red}\itshape,
  stringstyle=\color{darkgreen},
  showstringspaces=false,
  keywordstyle=\color{blue}\bfseries}

%%%%%%%%%%%%%%%%%%%%%%%%%%%%%%%%%%%%%%%%%%%%%%%%%%%%%%%%%%%%%%%%%%%%%%
% Macros
\setbeamercolor{emphbar}{bg=blue!20, fg=black}
\newcommand{\emphbar}[1]
{\begin{beamercolorbox}[rounded=true]{emphbar} 
      {#1}
 \end{beamercolorbox}
}
\newcounter{time}
\setcounter{time}{0}
\newcommand{\inctime}[1]{\addtocounter{time}{#1}{\tiny \thetime\ m}}

\newcommand{\typ}[1]{\lstinline{#1}}

\newcommand{\kwrd}[1]{ \texttt{\textbf{\color{blue}{#1}}}  }

%%%%%%%%%%%%%%%%%%%%%%%%%%%%%%%%%%%%%%%%%%%%%%%%%%%%%%%%%%%%%%%%%%%%%%
% Title page
\title[Using Linux Tools]{SEES: Using Linux Tools}

\author[FOSSEE] {FOSSEE}

\institute[IIT Bombay] {Department of Aerospace Engineering\\IIT Bombay}
\date[]{}
%%%%%%%%%%%%%%%%%%%%%%%%%%%%%%%%%%%%%%%%%%%%%%%%%%%%%%%%%%%%%%%%%%%%%%

%\pgfdeclareimage[height=0.75cm]{iitmlogo}{iitmlogo}
%\logo{\pgfuseimage{iitmlogo}}


%% Delete this, if you do not want the table of contents to pop up at
%% the beginning of each subsection:

\AtBeginSection[]
{
  \begin{frame}<beamer>
    \frametitle{Outline}
    \tableofcontents[currentsection]
  \end{frame}
}

% If you wish to uncover everything in a step-wise fashion, uncomment
% the following command: 
%\beamerdefaultoverlayspecification{<+->}

%%\includeonlyframes{current,current1,current2,current3,current4,current5,current6}

%%%%%%%%%%%%%%%%%%%%%%%%%%%%%%%%%%%%%%%%%%%%%%%%%%%%%%%%%%%%%%%%%%%%%%
% DOCUMENT STARTS
\begin{document}

\begin{frame}
  \maketitle
\end{frame}

% CREATING TOC 
\begin{frame}
  \frametitle{Outline}
  \tableofcontents
  % You might wish to add the option [pausesections]
\end{frame}


\section{Introduction}
\begin{frame}[fragile]
  \begin{block}{What is the Linux OS?}
    \begin{itemize}
    \item Free Open Source Operating System
      \begin{description}
        \item[Free] 
          Free as in Free Speech, not Free Beer
        \item[Open-Source]
          Permit modifications and redistribution of source code
      \end{description}
    \item Unix-inspired
    \item Linux Kernel + Application software
    \item Runs on a variety of hardware
    \item Also called GNU/Linux
    \end{itemize}
  \end{block}
\end{frame}

\begin{frame}[fragile]
  \frametitle{Why Linux?}
    \begin{itemize}
    \item Free as in Free Beer
    \item Secure \& versatile
    \end{itemize}

    \begin{block}{Why Linux for Scientific Computing?}
      \begin{itemize}
        \item Free as in Free Speech
        \item Can run for \emph{ever}
        \item Libraries
        \item Parallel Computing
      \end{itemize}
    \end{block}
\end{frame}

\section{Getting Started} 
\begin{frame}[fragile]
  \frametitle{Logging in}
  \begin{itemize}
  \item GNU/Linux does have a GUI
  \item Command Line for this module
  \item Hit \texttt{Ctrl + Alt + F1}
  \item Login 
  \item \texttt{logout} command logs you out
  \end{itemize}
\end{frame}

\begin{frame}[fragile]
  \frametitle{Where am I?}
  \begin{itemize}
  \item Logged in. Where are we?
  \item \texttt{pwd} command gives the present working directory
  \end{itemize} 
  \begin{lstlisting}
    $ pwd
    /home/user
  \end{lstlisting} % $
\end{frame}

\begin{frame}[fragile]
  \frametitle{What is in there?}
  \begin{itemize}
  \item \texttt{ls} command lists contents of \texttt{pwd}
  \end{itemize}
  \begin{lstlisting}
    $ ls
    jeeves.rst psmith.html blandings.html Music
  \end{lstlisting} %$
  \begin{itemize}
  \item Can also pass directory as argument 
  \end{itemize}
  \begin{lstlisting}
    $ ls Music
    one.mp3 two.mp3 three.mp3
  \end{lstlisting} %$
  \begin{itemize}
  \item   \alert{the Unix world is case sensitive}
  \end{itemize}
\end{frame}

\begin{frame}[fragile]
  \frametitle{New folders}
  \begin{itemize}
  \item \texttt{mkdir} creates new directories
  \end{itemize}
  \begin{lstlisting}
    $ mkdir sees
    $ ls
  \end{lstlisting}
  \begin{itemize}
  \item Special characters need to be escaped OR quoted
  \end{itemize}
  \begin{lstlisting}
    $ mkdir software\ engineering
    $ mkdir "software engg"
  \end{lstlisting} 
  \begin{itemize}
  \item Generally, use hyphens or underscores instead of spaces in names
  \end{itemize}
\end{frame}

\begin{frame}[fragile]
  \frametitle{Moving around}
  \begin{itemize}
  \item \texttt{cd} command changes the \texttt{pwd}
  \end{itemize}
  \begin{lstlisting}
    $ cd sees
    $ pwd 
    /home/user/sees/
  \end{lstlisting}
  \begin{itemize}
  \item Alternately written as \texttt{cd ./sees}
  \item Specifying path relative to \texttt{pwd}
  \item \texttt{..} takes one level up the directory structure
  \end{itemize}
  \begin{lstlisting}
    $ cd ..
  \end{lstlisting} % $
  \begin{itemize}
  \item We could use absolute path instead of relative
  \end{itemize}
  \begin{lstlisting}
    $ cd /home/user/sees/
  \end{lstlisting} % $
\end{frame}

\begin{frame}[fragile]
  \frametitle{New files}
  \begin{itemize}
  \item \texttt{touch} command creates a blank file
  \end{itemize}
  \begin{lstlisting}
    $ pwd
    /home/user
    $ cd sees
    $ touch first
    $ ls 
    first
  \end{lstlisting} % $
\end{frame}

\section{Getting Help}
\begin{frame}[fragile]
  \frametitle{What does a command do?}

  \begin{itemize}
  \item \texttt{whatis} gives a quick description of a command
  \end{itemize}
  \begin{lstlisting}
    $ whatis touch
    touch (1) - change file timestamps
  \end{lstlisting} % $
  \begin{itemize}
  \item \texttt{man} command gives more detailed description
  \end{itemize}
  \begin{lstlisting}
    $ man touch
  \end{lstlisting} % $
  \begin{itemize}
  \item Shows all tasks that the command can perform
  \item Hit \texttt{q} to quit the \texttt{man} page
  \item For more, see the \texttt{man} page of \texttt{man}
  \end{itemize}
  \begin{lstlisting}
    $ man man
  \end{lstlisting} % $
\end{frame}

\begin{frame}[fragile]
  \frametitle{Using additional options}

  \begin{itemize}
  \item \texttt{-h} or \texttt{--help} give summary of command usage
  \end{itemize}
  \begin{lstlisting}
    $ ls --help
  \end{lstlisting} % $
  \begin{itemize}
  \item List out all files within a directory, recursively
  \end{itemize}
  \begin{lstlisting}
    $ ls -R
  \end{lstlisting} % $
  \begin{itemize}
  \item Create a new directory along with parents, if required
  \end{itemize}
  \begin{lstlisting}
    $ pwd
    /home/user/
    $ ls sees/
    $ mkdir -p sees/linux-tools/scripts
  \end{lstlisting} % $
\end{frame}

\begin{frame}[fragile]
  \frametitle{Searching for a command}

  \begin{itemize}
  \item \texttt{apropos} searches commands based on their descriptions
  \end{itemize}

  \begin{lstlisting}
    $ apropos remove
  \end{lstlisting} % $

  \begin{itemize}
  \item Returns a list of all commands that contain the search term
  \item In this case, we are interested in \texttt{rm}, \texttt{rmdir}
  \end{itemize}
\end{frame}

\section{Basic File Handling}
\begin{frame}[fragile]
  \frametitle{Removing files}

  \begin{itemize}
  \item   \texttt{rm} is used to delete files
  \end{itemize}

  \begin{lstlisting}
    $ rm foo
  \end{lstlisting} % $

  \begin{itemize}
  \item \alert{\texttt{rm} works only for files; not directories}
  \end{itemize}

  \begin{itemize}
  \item Additional arguments required to remove a directory 
  \item \texttt{-r} stands for recursive. 
  \item Removes directory and all of it's content 
  \end{itemize}

  \begin{lstlisting}
    $ rm -r bar
  \end{lstlisting} % $

  \begin{itemize}
  \item \alert{\texttt{rmdir} can also be used; Explore} 
  \end{itemize}

\end{frame}


\begin{frame}[fragile]
  \frametitle{Copying Files}

  \begin{itemize}
  \item \texttt{cp} copies files from one location to another
  \end{itemize}

  \begin{lstlisting}
    $ cp linux-tools/scripts/foo linux-tools/
  \end{lstlisting} % $

  \begin{itemize}
  \item New file-name can be used at target location
  \item \texttt{foo} copied to new location with the name \texttt{bar} 
  \end{itemize}

  \begin{lstlisting}
    $ cp linux-tools/scripts/foo linux-tools/bar
  \end{lstlisting} % $

  \begin{itemize}
  \item \texttt{cp} overwrites files, unless explicitly asked not to
  \item To prevent this, use the \texttt{-i} flag 
  \end{itemize}
  
  \begin{lstlisting}
    $ cp -i linux-tools/scripts/foo linux-tools/bar
    cp: overwrite `bar'?
  \end{lstlisting} % $
\end{frame}

\begin{frame}[fragile]
  \frametitle{Copying Directories}

  \begin{itemize}
  \item \texttt{-r} is required to copy a directory and all it's
    content
  \item Copying directories is similar to copying files
  \end{itemize}

  \begin{lstlisting}
    $ cd /home/user
    $ cp -ir sees course
  \end{lstlisting}
\end{frame}



\begin{frame}[fragile]
  \frametitle{Moving Files}
  \begin{itemize}
  \item \texttt{cp} and \texttt{rm} would be one way
  \item \texttt{mv} command does the job
  \item Also takes \texttt{-i} option to prompt before overwriting
  \end{itemize}

  \begin{lstlisting}
    $ cd /home/user
    # Assume we have course directory already created
    $ mv -i sees/ course/
  \end{lstlisting}
  \begin{itemize}
  \item No prompt! Why?
  \end{itemize}
  \begin{lstlisting}
    $ ls course
  \end{lstlisting} % $
  \begin{itemize}
  \item \texttt{sees} became a sub-directory of \texttt{course} 
  \item \texttt{mv} command doesn't over-write directories
  \item \texttt{-i} option is useful when moving files around
  \item \texttt{mv} to rename --- move to same location with new name
  \end{itemize}
\end{frame}

\section{Linux File Hierarchy, Permissions \& Ownership}
\begin{frame}
  \frametitle{Linux File Hierarchy}
  \begin{itemize}
  \item \texttt{/} is called the root directory
  \item It is the topmost level of the hierarchy
  \item For details \texttt{man hier}
  \end{itemize}
\end{frame}

\begin{frame}[fragile]
  \frametitle{Permissions and Access control}

  \begin{itemize}
  \item In a multi-user environment, access control is vital
  \item Look at the output of \texttt{ls -l}
  \end{itemize}

  \begin{lstlisting}
    drwxr-xr-x   5 root users  4096 Jan 21 20:07 home
  \end{lstlisting} % $

  \begin{itemize}
  \item The first column shows the permission information
  \item First character specifies type of the file
  \item Files have \texttt{-}; Directories have \texttt{d}
  \item 3 sets of 3 characters --- for user, group and others
  \item \texttt{r}, \texttt{w}, \texttt{x} --- for read, write, execute
  \item Either the corresponding character or \texttt{-} is present
  \end{itemize}
\end{frame}

\begin{frame}[fragile]
  \frametitle{Changing the permissions}
  \begin{itemize}
  \item Permissions can be changed by owner of the file
  \item \texttt{chmod} command is used
  \item \texttt{-R} option to recursively change for all content of a
    directory
  \end{itemize}
  \begin{itemize}
  \item Change permissions of \texttt{foo.sh} from
    \texttt{-rw-r-{}-r-{}-} to \texttt{-rwxr-xr-{}-}
  \end{itemize}
  \begin{lstlisting}
    $ ls -l foo.sh
    $ chmod ug+x foo.sh
    $ ls -l foo.sh
  \end{lstlisting} % $
\end{frame}

\begin{frame}[fragile]
  \frametitle{Symbolic modes}
  \begin{small}
    \begin{center}
      \begin{tabular}{lll}
        Reference  &  Class   &  Description                                                      \\
        \hline
        u          &  user    &  the owner of the file                                            \\
        g          &  group   &  users who are members of the file's group                        \\
        o          &  others  &  users who are not hte owner of the file or members of the group  \\
        a          &  all     &  all three of the above; is the same as \emph{ugo}                \\
      \end{tabular}
    \end{center}

    \begin{center}
      \begin{tabular}{ll}
        Operator  &  Description                                                                   \\
        \hline
        +         &  adds the specified modes to the specified classes                             \\
        -         &  removes the specified modes from the specified classes                        \\
        =         &  the modes specified are to be made the exact modes for the specified classes  \\
      \end{tabular}
    \end{center}

    \begin{center}
      \begin{tabular}{lll}
        Mode  &  Name     &  Description                                 \\
        \hline
        r     &  read     &  read a file or list a directory's contents  \\
        w     &  write    &  write to a file or directory                \\
        x     &  execute  &  execute a file or recurse a directory tree  \\
      \end{tabular}
    \end{center}
  \end{small}
\end{frame}

\begin{frame}[fragile]
  \frametitle{Changing Ownership of Files}
  \begin{itemize}
  \item \texttt{chown} changes the owner and group
  \item By default, the user who creates file is the owner
  \item The default group is set as the group of the file
  \end{itemize}
  \begin{lstlisting}
    $ chown alice:users wonderland.txt
  \end{lstlisting} % $
  \begin{itemize}
  \item Did it work? \alert{Not every user can change ownership}
  \item Super-user or \texttt{root} user alone is empowered
  \end{itemize}
\end{frame}

\section{Looking at files}
\begin{frame}[fragile]
  \frametitle{\texttt{cat}}
  \begin{itemize}
  \item Displays the contents of files
  \end{itemize}
  \begin{lstlisting}
    $ cat foo.txt
  \end{lstlisting} % $
  \begin{itemize}
  \item Concatenates the text of multiple files
  \end{itemize}
  \begin{lstlisting}
    $ cat foo.txt bar.txt
  \end{lstlisting} % $
  \begin{itemize}
  \item Not-convenient to view long files
  \end{itemize}
\end{frame}

\begin{frame}[fragile]
  \frametitle{\texttt{less}}
  \begin{itemize}
  \item View contents of a file one screen at a time
  \end{itemize}

  \begin{lstlisting}
    $ less wonderland.txt
  \end{lstlisting} % $

  \begin{itemize}
  \item q: Quit
  \item Arrows/Page Up/Page Down/Home/End: Navigation
  \item ng: Jump to line number n
  \item /pattern: Search. Regular expressions can be used
  \item h: Help
  \end{itemize}
\end{frame}

\begin{frame}[fragile]
  \frametitle{\texttt{wc}}
  \begin{itemize}
  \item Statistical information about the file
  \item the number of lines in the file
  \item the number of words 
  \item the number of characters
  \end{itemize}

  \begin{lstlisting}
    $ wc wonderland.txt
  \end{lstlisting} % $
\end{frame}

\begin{frame}[fragile]
  \frametitle{\texttt{head} \& \texttt{tail}}
  \begin{itemize}
  \item let you see parts of files, instead of the whole file
  \item \texttt{head} -- start of a file; \texttt{tail} -- end of a
    file 
  \item show 10 lines by default
  \end{itemize}
  \begin{lstlisting}
    $ head wonderland.txt
  \end{lstlisting} % $
  \begin{itemize}
  \item \texttt{-n} option to change the number of lines
  \end{itemize}
  \begin{lstlisting}
    $ head -n 1 wonderland.txt
  \end{lstlisting} % $
  \begin{itemize}
  \item \texttt{tail} is commonly used to monitor files
  \item \texttt{-f} option to monitor the file
  \item \texttt{Ctrl-C} to interrupt
  \end{itemize}
  \begin{lstlisting}
    $ tail -f /var/log/dmesg
  \end{lstlisting} % $
\end{frame}


\begin{frame}[fragile]
  \frametitle{\texttt{cut}}
  \begin{itemize}
  \item Allows you to view only certain sections of lines
  \item Let's take \texttt{/etc/passwd} as our example 
  \end{itemize}
  \begin{lstlisting}
    root:x:0:0:root:/root:/bin/bash
  \end{lstlisting}
  \begin{itemize}
  \item View only user names of all the users (first column)
  \end{itemize}
  \begin{lstlisting}
    $ cut -d : -f 1 /etc/passwd
  \end{lstlisting} % $
  \begin{itemize}
  \item \texttt{-d} specifies delimiter between fields (default TAB)
  \item \texttt{-f} specifies the field number
  \item Multiple fields by separating field numbers with comma
  \end{itemize}
  \begin{lstlisting}
    $ cut -d : -f 1,5,7 /etc/passwd
  \end{lstlisting} % $
\end{frame}

\begin{frame}[fragile]
  \frametitle{\texttt{cut}}
  \begin{itemize}
  \item Allows choosing on the basis of characters or bytes
  \item Example below gets first 4 characters of \texttt{/etc/passwd}
  \end{itemize}
  \begin{lstlisting}
    $ cut -c 1-4 /etc/passwd
  \end{lstlisting} % $
  \begin{itemize}
  \item One of the limits of the range can be dropped
  \item Sensible defaults are assumed in such cases
  \end{itemize}
  \begin{lstlisting}
    $ cut -c -4 /etc/passwd
    $ cut -c 10- /etc/passwd
  \end{lstlisting} % $
\end{frame}

\begin{frame}[fragile]
  \frametitle{\texttt{paste}}
  \begin{itemize}
  \item Joins corresponding lines from two different files
    \begin{center}
      \begin{tabular}{l|l}
        \verb~students.txt~  &  \verb~marks.txt~  \\
        Hussain              &  89 92 85          \\
        Dilbert              &  98 47 67          \\
        Anne                 &  67 82 76          \\
        Raul                 &  78 97 60          \\
        Sven                 &  67 68 69          \\
      \end{tabular}
    \end{center}
  \end{itemize}
  \begin{lstlisting}
    $ paste students.txt marks.txt 
    $ paste -s students.txt marks.txt
  \end{lstlisting}
  \begin{itemize}
  \item \texttt{-s} prints content, one below the other
  \item If first column of marks file had roll numbers? How do we get
    a combined file with the same output as above (i.e. without roll
    numbers). We need to use \texttt{cut} \& \texttt{paste} together.
    But how?
  \end{itemize}
\end{frame}

\section{The Command Shell}

\begin{frame}[fragile]
  \frametitle{Redirection and Piping}
  \begin{lstlisting}
    $ cut -d " " -f 2- marks1.txt \
    > /tmp/m_tmp.txt
    $ paste -d " " students.txt m_tmp.txt
  \end{lstlisting} % $

  or

  \begin{lstlisting}
    $ cut -d " " -f 2- marks1.txt \
    | paste -d " " students.txt -
  \end{lstlisting} % $

  \begin{itemize}
  \item The first solution used Redirection
  \item The second solution uses Piping
  \end{itemize}
\end{frame}

\begin{frame}[fragile]
  \frametitle{Redirection} 

  \begin{itemize}
  \item The standard output (stdout) stream goes to the display
  \item Not always, what we need
  \item First solution, redirects output to a file
  \item \texttt{>} states that output is redirected; It is
    followed by location to redirect
  \end{itemize}
  \begin{lstlisting}
    $ command > file1
  \end{lstlisting} % $
  \begin{itemize}
  \item \texttt{>} creates a new file at specified location
  \item \texttt{>>} appends to a file at specified location
  \end{itemize}
\end{frame}

\begin{frame}[fragile]
  \frametitle{Redirection \ldots} 
  \begin{itemize}
  \item Similarly, the standard input (stdin) can be redirected
  \end{itemize}
  \begin{lstlisting}
    $ command < file1
  \end{lstlisting} % $
  \begin{itemize}
  \item input and the output redirection could be combined
  \end{itemize}
  \begin{lstlisting}
    $ command < infile > outfile
  \end{lstlisting} % $
  \begin{itemize}
  \item Standard error (stderr) is the third standard stream
  \item All error messages come through this stream
  \item \texttt{stderr} can also be redirected
  \end{itemize}
\end{frame}

\begin{frame}[fragile]
  \frametitle{Redirection \ldots} 
  \begin{itemize}
  \item Following example shows \texttt{stderr} redirection
  \item Error is printed out in the first case
  \item Error message is redirected, in the second case
  \end{itemize}
  \begin{lstlisting}
    $ cut -d " " -c 2- marks1.txt \
      > /tmp/m_tmp.txt

    $ cut -d " " -f 2- marks1.txt 1> \
      /tmp/m_tmp.txt 2> /tmp/m_err.txt
  \end{lstlisting} % $
  \begin{itemize}
  \item \texttt{1>} redirects \texttt{stdout}; \texttt{2>} redirects
    \texttt{stderr} 
  \end{itemize}
  \begin{lstlisting}
    $ paste -d " " students.txt m_tmp.txt
  \end{lstlisting} % $
\end{frame}

\begin{frame}[fragile]
  \frametitle{Piping}
  \begin{lstlisting}
    $ cut -d " " -f 2- marks1.txt \
      | paste -d " " students.txt -
  \end{lstlisting} % $
  \begin{itemize}
  \item \texttt{-} instead of FILE asks \texttt{paste} to read from
    \texttt{stdin}
  \item \texttt{cut} command is a normal command
  \item the \texttt{|} seems to be joining the two commands
  \item Redirects output of first command to \texttt{stdin}, which
    becomes input to the second command
  \item This is called piping; \texttt{|} is called a pipe
  \end{itemize}
\end{frame}

\begin{frame}[fragile]
  \frametitle{Piping}
  \begin{itemize}
  \item Roughly same as -- 2 redirects and a temporary file
  \end{itemize}
  \begin{lstlisting}
    $ command1 > tempfile
    $ command2 < tempfile
    $ rm tempfile
  \end{lstlisting} % $
  \begin{itemize}
  \item Any number of commands can be piped together
  \end{itemize}
\end{frame}

\subsection{Features of the Shell}

\begin{frame}[fragile]
  \frametitle{Tab-completion}
  \begin{itemize}
  \item Hit tab to complete an incompletely typed word
  \item Tab twice to list all possibilities when ambiguous completion
  \item Bash provides tab completion for the following.
    \begin{enumerate}
    \item File Names
    \item Directory Names
    \item Executable Names
    \item User Names (when they are prefixed with a \~{})
    \item Host Names (when they are prefixed with a @)
    \item Variable Names (when they are prefixed with a \$)
    \end{enumerate}
  \end{itemize}
\end{frame}

\begin{frame}[fragile]
  \frametitle{History}
  \begin{itemize}
  \item Bash saves history of commands typed
  \item Up and down arrow keys allow to navigate history
  \item \texttt{Ctrl-r} searches for commands used
  \end{itemize}
\end{frame}

\begin{frame}[fragile]
  \frametitle{Shell Meta Characters}
  \begin{itemize}
  \item ``meta characters''  are special command directives
  \item File-names shouldn't have meta-characters
  \item   \verb+/<>!$%^&*|{}[]"'`~;+
  \end{itemize}
  \begin{lstlisting}
    $ ls file.*
  \end{lstlisting} % $
  \begin{itemize}
  \item Lists \texttt{file.ext} files, where \texttt{ext} can be
    anything
  \end{itemize}
  \begin{lstlisting}
    $ ls file.?
  \end{lstlisting} % $
  \begin{itemize}
  \item Lists \texttt{file.ext} files, where \texttt{ext} is only one
    character 
  \end{itemize}
\end{frame}

\section{More text processing}

\begin{frame}[fragile]
  \frametitle{\texttt{sort}}
  \begin{itemize}
  \item \texttt{sort} can be used to get sorted content
  \item Command below prints student marks, sorted by name
  \end{itemize}
  \begin{lstlisting}
    $ cut -d " " -f 2- marks1.txt \
      | paste -d " " students.txt - \
      | sort
  \end{lstlisting} % $
  \begin{itemize}
  \item The default is sort based on the whole line
  \item \texttt{sort} can sort based on a particular field
  \end{itemize}
\end{frame}

\begin{frame}[fragile]
  \frametitle{\texttt{sort} \ldots}
  \begin{itemize}
  \item The command below sorts based on marks in first subject
  \end{itemize}
  \begin{lstlisting}
    $ cut -d " " -f 2- marks1.txt \
      | paste -d " " students.txt -\
      | sort -t " " -k 2 -rn
  \end{lstlisting} % $
  \begin{itemize}
  \item \texttt{-t} specifies the delimiter between fields
  \item \texttt{-k} specifies the field to use for sorting
  \item \texttt{-n} to choose numerical sorting
  \item \texttt{-r} for sorting in the reverse order
  \end{itemize}
\end{frame}

\begin{frame}[fragile]
  \frametitle{\texttt{grep}}
  \begin{itemize}
  \item \texttt{grep} is a command line text search utility
  \item Command below searches \& shows the marks of Anne alone 
  \end{itemize}
  \begin{lstlisting}
    $ cut -d " " -f 2- marks1.txt \
    | paste -d " " students.txt - 
    | grep Anne
  \end{lstlisting} % $
  \begin{itemize}
  \item \texttt{grep} is case-sensitive by default
  \end{itemize}
\end{frame}

\begin{frame}[fragile]
  \frametitle{\texttt{grep} \ldots}
  \begin{itemize}
  \item \texttt{-i} for case-insensitive searches
  \end{itemize}
  \begin{lstlisting}
    $ cut -d " " -f 2- marks1.txt \
    | paste -d " " students.txt - 
    | grep -i Anne
  \end{lstlisting} % $
  \begin{itemize}
  \item \texttt{-v} inverts the search
  \item To see everyone's marks except Anne's
  \end{itemize}
  \begin{lstlisting}
    $ cut -d " " -f 2- marks1.txt \
    | paste -d " " students.txt - 
    | grep -iv Anne
  \end{lstlisting} % $
\end{frame}

\begin{frame}[fragile]
  \frametitle{\texttt{tr}}
  \begin{itemize}
  \item \texttt{tr} translates or deletes characters
  \item Reads from \texttt{stdin} and outputs to \texttt{stdout}
  \item Given, two sets of characters, replaces one with other
  \item The following, replaces all lower-case with upper-case 
  \end{itemize}
  \begin{lstlisting}
    $ cat students.txt | tr a-z A-Z
  \end{lstlisting} % $
  \begin{itemize}
  \item \texttt{-s} compresses sequences of identical adjacent
    characters in the output to a single one
  \item Following command removes empty newlines
  \end{itemize}
  \begin{lstlisting}
    $ tr -s '\n' '\n'
  \end{lstlisting} % $
\end{frame}

\begin{frame}[fragile]
  \frametitle{\texttt{tr} \ldots}
  \begin{itemize}
  \item \texttt{-d} deletes all specified characters
  \item Only a single character set argument is required
  \item The following command removes carriage return characters
    (converting file in DOS/Windows format to the Unix format)
  \end{itemize}
  \begin{lstlisting}
    $ cat foo.txt | tr -d '\r' > bar.txt
  \end{lstlisting} % $
  \begin{itemize}
  \item \texttt{-c} complements the first set of characters
  \item The following command removes all non-alphanumeric characters
  \end{itemize}
  \begin{lstlisting}
    $ tr -cd '[:alnum:]'
  \end{lstlisting} % $
\end{frame}

\begin{frame}[fragile]
  \frametitle{\texttt{uniq}}
  \begin{itemize}
  \item \texttt{uniq} command removes duplicates from \alert{sorted} input
  \end{itemize}
  \begin{lstlisting}
    $ sort items.txt | uniq
  \end{lstlisting} % $
  \begin{itemize}
  \item \texttt{uniq -u} gives lines which do not have any duplicates
  \item \texttt{uniq -d} outputs only those lines which have duplicates
  \item \texttt{-c} displays the number of times each line occurs 
  \end{itemize}
  \begin{lstlisting}
    $ sort items.txt | uniq -u 
    $ sort items.txt | uniq -dc
  \end{lstlisting} % $
\end{frame}

\section{Simple Shell Scripts}

\begin{frame}[fragile]
  \frametitle{Shell scripts}
  \begin{itemize}
  \item Simply a sequence of shell commands in a file
  \item To save results of students in \texttt{results.txt} in
    \texttt{marks} dir
  \end{itemize}
  \begin{lstlisting}
    #!/bin/bash
    mkdir ~/marks
    cut -d " " -f 2- marks1.txt \
    | paste -d " " students.txt - \
    | sort > ~/marks/results.txt
  \end{lstlisting} % $
\end{frame}

\begin{frame}[fragile]
  \frametitle{Shell scripts \ldots}
  \begin{itemize}
  \item Save the script as \texttt{results.sh}
  \item Make file executable and then run
  \end{itemize}
  \begin{lstlisting}
    $ chmod u+x results.sh
    $ ./results.sh
  \end{lstlisting} % $
  \begin{itemize}
  \item What does the first line in the script do?
  \item Specify the interpreter or shell which should be used to
    execute the script; in this case \texttt{bash}
  \end{itemize}
\end{frame}

\begin{frame}[fragile]
  \frametitle{Variables \& Comments}
  \begin{lstlisting}
    $ name=FOSSEE
    $ count=`wc -l wonderland.txt`
    $ echo $count # Shows the value of count
  \end{lstlisting} % $
  \begin{itemize}
  \item It is possible to create variables in shell scripts
  \item Variables can be assigned with the output of commands
  \item \alert{NOTE:} There is no space around the \texttt{=} sign
  \item All text following the \texttt{\#} is considered a comment
  \end{itemize}
\end{frame}

\begin{frame}[fragile]
  \frametitle{\texttt{echo}}
  \begin{itemize}
  \item \texttt{echo} command prints out messages
  \end{itemize}
  \begin{lstlisting}
    #!/bin/bash
    mkdir ~/marks
    cut -d " " -f 2- marks1.txt \
    | paste -d " " students.txt - \
    | sort > ~/marks/results.txt
    echo "Results generated."
  \end{lstlisting} % $
\end{frame}

\begin{frame}[fragile]
  \frametitle{Command line arguments}
  \begin{itemize}
  \item Shell scripts can be given command line arguments
  \item Following code allows to specify the results file
  \end{itemize}
  \begin{lstlisting}
    #!/bin/bash
    mkdir ~/marks
    cut -d " " -f 2- marks1.txt \
    | paste -d " " students.txt - \
    | sort > ~/marks/$1
    echo "Results generated."
  \end{lstlisting} % $
  \begin{itemize}
  \item \texttt{\$1} corresponds to first command line argument 
  \item \texttt{\$n} corresponds to $n{th}$ command line argument
  \item It can be run as shown below
  \end{itemize}
  \begin{lstlisting}
    $ ./results.sh grades.txt
  \end{lstlisting} % $
\end{frame}

\begin{frame}[fragile]
  \frametitle{\texttt{PATH}}
  \begin{itemize}
  \item The shell searches in a set of locations, for the command 
  \item Locations are saved in ``environment'' variable called PATH
  \item \texttt{echo} can show the value of variables
  \end{itemize}
  \begin{lstlisting}
    $ echo $PATH
  \end{lstlisting} % $
  \begin{itemize}
  \item Put \texttt{results.sh} in one of these locations
  \item It can then be run without \texttt{./} 
  \end{itemize}
\end{frame}

\section{Control structures and Operators}
\begin{frame}[fragile]
  \frametitle{Control Structures}
  \begin{itemize}
  \item \texttt{if-else}
  \item \texttt{for} loops
  \item \texttt{while} loops
  \end{itemize}
  \begin{itemize}
  \item \texttt{test} command to test for conditions
  \item A whole range of tests that can be performed
    \begin{itemize}
    \item \texttt{STRING1 = STRING2} -- string equality
    \item \texttt{INTEGER1 -eq INTEGER2} -- integer equality
    \item \texttt{-e FILE} -- existence of FILE 
    \end{itemize}
  \item \texttt{man} page of \texttt{test} gives list of various tests
  \end{itemize}
\end{frame}

\begin{frame}[fragile]
  \frametitle{\texttt{if}}
  \begin{itemize}
  \item Print message if directory exists in \texttt{pwd}
  \end{itemize}
  \begin{lstlisting}
    #!/bin/bash
    if test -d $1
    then
    echo "Yes, the directory" \
    $1 "is present"
    fi
  \end{lstlisting} % $
\end{frame}

\begin{frame}[fragile]
  \frametitle{\texttt{if}-\texttt{else}}
  \begin{itemize}
  \item Checks whether argument is negative or not
  \end{itemize}
  \begin{lstlisting}
    #!/bin/bash
    if test $1 -lt 0
    then
    echo "number is negative"
    else
    echo "number is non-negative"
    fi
  \end{lstlisting} % $
  \begin{lstlisting}
    $ ./sign.sh -11
  \end{lstlisting} % $
\end{frame}

\begin{frame}[fragile]
  \frametitle{\texttt{[ ]} - alias for \texttt{test}}
  \begin{itemize}
  \item Square brackets (\texttt{[]}) can be used instead of
    \texttt{test}
  \item 
  \end{itemize}
  \begin{lstlisting}
    #!/bin/bash
    if [ $1 -lt 0 ]
    then
    echo "number is negative"
    else
    echo "number is non-negative"
    fi
  \end{lstlisting} % $
  \begin{itemize}
  \item \alert{spacing is important, when using the square brackets}
  \end{itemize}
\end{frame}

\begin{frame}[fragile]
  \frametitle{\texttt{if}-\texttt{else}}
  \begin{itemize}
  \item An example script to greet the user, based on the time
  \end{itemize}
  \begin{lstlisting}
    #!/bin/sh
    # Script to greet the user 
    # according to time of day
    hour=`date | cut -c12-13`
    now=`date +"%A, %d of %B, %Y (%r)"`
    if [ $hour -lt 12 ]
    then
    mess="Good Morning \
    $LOGNAME, Have a nice day!"
    fi
  \end{lstlisting} %$
\end{frame}

\begin{frame}[fragile]
  \frametitle{\texttt{if}-\texttt{else} \ldots}
  \begin{lstlisting}
    if [ $hour -gt 12 -a $hour -le 16 ]
    then
    mess="Good Afternoon $LOGNAME"
    fi
    if [ $hour -gt 16 -a $hour -le 18 ]
    then
    mess="Good Evening $LOGNAME"
    fi
    echo -e "$mess\nIt is $now"
  \end{lstlisting} % $
  \begin{itemize}
  \item \texttt{\$LOGNAME} has login name (env. variable) 
  \item backquotes store commands outputs into variables
  \end{itemize}
\end{frame}

\begin{frame}[fragile]
  \frametitle{\texttt{for}}
  \begin{block}{Problem}
    Given a set of \texttt{.mp3} files, that have names beginning with
    numbers followed by their names --- \texttt{08 - Society.mp3} ---
    rename the files to have just the names. Also replace any spaces
    in the name with hyphens. 
  \end{block}
  \begin{itemize}
  \item Loop over the list of files
  \item Process the names, to get new names
  \item Rename the files
  \end{itemize}
\end{frame}

\begin{frame}[fragile]
  \frametitle{\texttt{for}} 
  \begin{itemize}
  \item A simple example of the \texttt{for} loop
  \end{itemize}
  \begin{lstlisting}
    for animal in rat cat dog man
    do 
    echo $animal
    done
  \end{lstlisting} % $
  \begin{itemize}
  \item List of animals, each animal's name separated by a space
  \item Loop over the list; \texttt{animal} is a dummy variable
  \item Echo value of \texttt{animal} ---  each name in list
  \end{itemize}
  \begin{lstlisting}
    for i in {10..20}
    do 
    echo $i
    done
  \end{lstlisting} % $
\end{frame}

\begin{frame}[fragile]
  \frametitle{\texttt{for}}
  \begin{itemize}
  \item Let's start with echoing the names of the files
  \end{itemize}
  \begin{lstlisting}
    for i in `ls *.mp3`
    do
    echo "$i"
    done
  \end{lstlisting} % $
  \begin{itemize}
  \item Spaces in names cause trouble!
  \item The following works better
  \end{itemize}
  \begin{lstlisting}
    for i in *.mp3
    do
    echo "$i"
    done
  \end{lstlisting} % $
\end{frame}

\begin{frame}[fragile]
  \frametitle{\texttt{tr} \& \texttt{cut}}
  \begin{itemize}
  \item Replace all spaces with hyphens using \texttt{tr -s}
  \item Use cut \& keep only the text after the first hyphen
  \end{itemize}
  \begin{lstlisting}
  for i in *.mp3
  do 
  echo $i|tr -s " " "-"|cut -d - -f 2-
  done
  \end{lstlisting} % $
  Now \texttt{mv}, instead of just echoing
  \begin{lstlisting}
    for i in *.mp3
    do 
    mv $i `echo $i|tr -s " " "-"\
    |cut -d - -f 2-`
    done
  \end{lstlisting} % $
\end{frame}


\begin{frame}[fragile]
  \frametitle{\texttt{while}}
  \begin{itemize}
  \item Continuously execute a block of commands until condition
    becomes false
  \end{itemize}

  \begin{itemize}
  \item program that takes user input and prints it back, until the
    input is \texttt{quit}
  \end{itemize}

  \begin{lstlisting}
    while [ "$variable" != "quit" ]
    do
    read variable
    echo "Input - $variable"
    done
    exit 0
  \end{lstlisting} % $
\end{frame}

\begin{frame}[fragile]
  \frametitle{Environment Variables}
  \begin{itemize}
  \item Pass information from shell to programs running in it
  \item Behavior of programs can change based on values of variables
  \item Environment variables vs. Shell variables
  \item Shell variables -- only current instance of the shell
  \item Environment variables -- valid for the whole session
  \item Convention -- environment variables are UPPER CASE
  \end{itemize}
  \begin{lstlisting}
    $ echo $OSTYPE 
    linux-gnu
    $ echo $HOME
    /home/user
  \end{lstlisting} % $
\end{frame}

\begin{frame}[fragile]
  \frametitle{Environment Variables \ldots}
  \begin{itemize}
  \item The following commands show values of all the environment
    variables
  \end{itemize}
  \begin{lstlisting}
    $ printenv | less
    $ env
  \end{lstlisting} % $
  \begin{itemize}
  \item Use \texttt{export} to change Environment variables
  \item The new value is available to all programs started from the shell
  \end{itemize}
  \begin{lstlisting}
    $ export PATH=$PATH:$HOME/bin
  \end{lstlisting} % $
\end{frame}

\section{Miscellaneous Tools}

\begin{frame}[fragile]
  \frametitle{\texttt{find}}
  \begin{itemize}
  \item Find files in a directory hierarchy
  \item Offers a very complex feature set
  \item Look at the \texttt{man} page!
  \end{itemize}
  \begin{itemize}
  \item Find all \texttt{.pdf} files, in current dir and sub-dirs
    \begin{lstlisting}
      $ find . -name ``*.pdf''
    \end{lstlisting} % $
  \item List all the directory and sub-directory names
    \begin{lstlisting}
      $ find . -type d
    \end{lstlisting} % $
  \end{itemize}
\end{frame}

\begin{frame}[fragile]
  \frametitle{\texttt{cmp}}
  \begin{itemize}
  \item Compare two files
  \end{itemize}
  \begin{lstlisting}
    $ find . -name quick.c
    ./Desktop/programs/quick.c
    ./c-folder/quick.c
    $ cmp Desktop/programs/quick.c \
    c-folder/quick.c
  \end{lstlisting} % $
  \begin{itemize}
  \item No output when the files are exactly the same
  \item Else, gives location where the first difference occurs 
  \end{itemize}
\end{frame}

\begin{frame}[fragile]
  \frametitle{\texttt{diff}}
  \begin{itemize}
  \item We know the files are different, but want exact differences
  \end{itemize}
  \begin{lstlisting}
    $ diff Desktop/programs/quick.c \
    c-folder/quick.c
  \end{lstlisting} % $
  \begin{itemize}
  \item line by line difference between files
  \item \texttt{>} indicates content only in second file
  \item \texttt{<} indicates content only in first file
  \end{itemize}
\end{frame}

\begin{frame}[fragile]
\frametitle{\texttt{tar}}
\begin{itemize}
\item \emph{tarball} -- essentially a collection of files
\item May or may not be compressed
\item Eases the job of storing, backing-up \& transporting files
\end{itemize}
\end{frame}

\begin{frame}[fragile]
\frametitle{Extracting an archive}

\begin{lstlisting}
$ mkdir extract 
$ cp allfiles.tar extract/ 
$ cd extract 
$ tar -xvf allfiles.tar
\end{lstlisting} %$

\begin{itemize}
\item \texttt{-x} --- Extract files within the archive
\item \texttt{-f} --- Specify the archive file
\item \texttt{-v} --- Be verbose
\end{itemize}
\end{frame}

\begin{frame}[fragile]
\frametitle{Creating an archive}
\begin{lstlisting}
$ tar -cvf newarchive.tar *.txt
\end{lstlisting} % $
\begin{itemize}
\item \texttt{-c} --- Create archive
\item Last argument is list of files to be added to archive
\end{itemize}
\end{frame}

\begin{frame}[fragile]
  \frametitle{Compressed archives}
  \begin{itemize}
  \item \texttt{tar} can create and extract compressed archives
  \item Supports compressions like gzip, bzip2, lzma, etc.
  \item Additional option to handle compressed archives
    \begin{center}
      \begin{tabular}{ll}
        Compression      &  Option   \\
        gzip   &  \texttt{-z}        \\
        bzip2  &  \texttt{-j}        \\
        lzma   &  \texttt{-{}-lzma}  \\
      \end{tabular}
    \end{center}
  \end{itemize}
  \begin{lstlisting}
    $ tar -cvzf newarchive.tar.gz *.txt
  \end{lstlisting} % $
\end{frame}


\begin{frame}
\frametitle{Customizing your shell}
\begin{itemize}
\item Bash reads \texttt{/etc/profile},
  \texttt{\textasciitilde{}/.bash\_profile},
  \texttt{\textasciitilde{}/.bash\_login}, and
  \texttt{\textasciitilde{}/.profile} in that order, when starting
  up as a login shell. 
\item \texttt{\textasciitilde{}/.bashrc} is read, when not a login
  shell 
\item Put any commands that you want to run when bash starts, in this
  file. 
\end{itemize}
\end{frame}

%% THE DOCUMENT ENDS HERE
\end{document}
%%%%%%%%%%%%%%%%%%%%

\section{Basic editing and editors}
\begin{frame}[fragile]
\frametitle{vim}


Vim is a very powerful editor. It has a lot of commands, and all of them
cannot be explained here. We shall try and look at a few, so that you
can find your way around in vim.

To open a file in vim, we pass the filename as a parameter to the \texttt{vim}
command. If a file with that filename does not exist, a new file is
created.

\begin{lstlisting}
$ vim first.txt
\end{lstlisting} % $

To start inserting text into the new file that we have opened, we need
to press the \texttt{i} key. This will take us into the \emph{insert} mode from the
\emph{command} mode. Hitting the \texttt{esc} key, will bring us back to the
\emph{command} mode. There is also another mode of vim, called the \emph{visual}
mode which will be discussed later in the course.

In general, it is good to spend as little time as possible in the insert
mode and extensively use the command mode to achieve various tasks.

To save the file, use \texttt{:w} in the command mode. From here on, it is
understood that we are in the command mode, whenever we are issuing any
command to vim.

To save a file and continue editing, use \texttt{:w FILENAME} The file name is
optional. If you do not specify a filename, it is saved in the same file
that you opened. If a file name different from the one you opened is
specified, the text is saved with the new name, but you continue editing
the file that you opened. The next time you save it without specifying a
name, it gets saved with the name of the file that you initially opened.

To save file with a new name and continue editing the new file, use
\texttt{:saveas FILENAME}

To save and quit, use \texttt{:wq}

To quit, use \texttt{:q}

To quit without saving, use \texttt{:q!}
\begin{itemize}

\item Moving around\\
While you are typing in a file, it is in-convenient to keep moving your
fingers from the standard position for typing to the arrow keys. Vim,
therefore, provides alternate keys for moving in the document. Note
again that, you should be in the command mode, when issuing any commands
to vim.

The basic cursor movement can be achieved using the keys, \texttt{h} (left),
\texttt{l} (right), \texttt{k} (up) and \texttt{j} (down).

\begin{lstlisting}
^
k
\end{lstlisting} % $

\begin{quote}

\begin{description}
\item[< h l >] j v
\end{description}

\end{quote}

Note: Most commands can be prefixed with a number, to repeat the
command. For instance, \texttt{10j} will move the cursor down 10 lines.


\item Moving within a line\\
\begin{center}
\begin{tabular}{ll}
 Cursor Movement                    &  Command                      \\
\hline
 Beginning of line                  &  \texttt{0}                   \\
 First non-space character of line  &  \texttt{\textasciicircum{}}  \\
 End of line                        &  \texttt{\$}                  \\
 Last non-space character of line   &  \texttt{g\_}                 \\
\end{tabular}
\end{center}




\item Moving by words and sentences\\
\begin{center}
\begin{tabular}{ll}
 Cursor Movement                &  Command      \\
\hline
 Forward, word beginning        &  \texttt{w}   \\
 Backward, word beginning       &  \texttt{b}   \\
 Forward, word end              &  \texttt{e}   \\
 Backward, word end             &  \texttt{ge}  \\
 Forward, sentence beginning    &  \texttt{)}   \\
 Backward, sentence beginning   &  \texttt{(}   \\
 Forward, paragraph beginning   &  \texttt{\}}  \\
 Backward, paragraph beginning  &  \texttt{\{}  \\
\end{tabular}
\end{center}




\item More movement commands\\
\begin{center}
\begin{tabular}{ll}
 Cursor Movement                                   &  Command          \\
\hline
 Forward by a screenful of text                    &  \texttt{C-f}     \\
 Backward by a screenful of text                   &  \texttt{C-b}     \\
 Beginning of the screen                           &  \texttt{H}       \\
 Middle of the screen                              &  \texttt{M}       \\
 End of the screen                                 &  \texttt{L}       \\
 End of file                                       &  \texttt{G}       \\
 Line number \texttt{num}                          &  \texttt{[num]G}  \\
 Beginning of file                                 &  \texttt{gg}      \\
 Next occurrence of the text under the cursor      &  \texttt{*}       \\
 Previous occurrence of the text under the cursor  &  \texttt{\#}      \\
\end{tabular}
\end{center}



Note: \texttt{C-x} is \texttt{Ctrl} + \texttt{x}


\item The visual mode\\
The visual mode is a special mode that is not present in the original vi
editor. It allows us to highlight text and perform actions on it. All
the movement commands that have been discussed till now work in the
visual mode also. The editing commands that will be discussed in the
future work on the visual blocks selected, too.


\item Editing commands\\
The editing commands usually take the movements as arguments. A movement
is equivalent to a selection in the visual mode. The cursor is assumed
to have moved over the text in between the initial and the final points
of the movement. The motion or the visual block that's been highlighted
can be passed as arguments to the editing commands.


\begin{center}
\begin{tabular}{ll}
 Editing effect           &  Command     \\
\hline
 Cutting text             &  \texttt{d}  \\
 Copying/Yanking text     &  \texttt{y}  \\
 Pasting copied/cut text  &  \texttt{p}  \\
\end{tabular}
\end{center}



The cut and copy commands take the motions or visual blocks as arguments
and act on them. For instance, if you wish to delete the text from the
current text position to the beginning of the next word, type \texttt{dw}. If
you wish to copy the text from the current position to the end of this
sentence, type \texttt{y)}.

Apart from the above commands, that take any motion or visual block as
an argument, there are additional special commands.


\begin{center}
\begin{tabular}{ll}
 Editing effect                                          &  Command      \\
\hline
 Cut the character under the cursor                      &  \texttt{x}   \\
 Replace the character under the cursor with \texttt{a}  &  \texttt{ra}  \\
 Cut an entire line                                      &  \texttt{dd}  \\
 Copy/yank an entire line                                &  \texttt{yy}  \\
\end{tabular}
\end{center}



Note: You can prefix numbers to any of the commands, to repeat them.


\item Undo and Redo\\
You can undo almost anything using \texttt{u}.

To undo the undo command type \texttt{C-r}


\item Searching and Replacing\\
\begin{center}
\begin{tabular}{ll}
 Finding                                        &  Command                        \\
\hline
 Next occurrence of \texttt{text}, forward      &  \texttt{\textbackslash{}text}  \\
 Next occurrence of \texttt{text}, backward     &  \texttt{?text}                 \\
 Search again in the same direction             &  \texttt{n}                     \\
 Search again in the opposite direction         &  \texttt{N}                     \\
 Next occurrence of \texttt{x} in the line      &  \texttt{fx}                    \\
 Previous occurrence of \texttt{x} in the line  &  \texttt{Fx}                    \\
\end{tabular}
\end{center}




\begin{center}
\begin{tabular}{ll}
 Finding and Replacing                                                                                             &  Command                   \\
\hline
 Replace the first instance of \texttt{old} with \texttt{new} in the current line.                                 &  \texttt{:s/old/new}       \\
 Replace all instances of \texttt{old} with \texttt{new} in the current line.                                      &  \texttt{:s/old/new/g}     \\
 Replace all instances of \texttt{old} with \texttt{new} in the current line, but ask for confirmation each time.  &  \texttt{:s/old/new/gc}    \\
 Replace the first instance of \texttt{old} with \texttt{new} in the entire file.                                  &  \texttt{:\%s/old/new}     \\
 Replace all instances of \texttt{old} with \texttt{new} in the entire file.                                       &  \texttt{:\%s/old/new/g}   \\
 Replace all instances of \texttt{old} with \texttt{new} in the entire file but ask for confirmation each time.    &  \texttt{:\%s/old/new/gc}  \\
\end{tabular}
\end{center}



\end{itemize} % ends low level
\end{frame}
\begin{frame}
\frametitle{SciTE}


SciTE is a \emph{source code} editor, that has a feel similar to the commonly
used GUI text editors. It has a wide range of features that are
extremely useful for a programmer, editing code. Also it aims to keep
configuration simple, and the user needs to edit a text file to
configure SciTE to his/her liking.

Opening, Saving, Editing files with SciTE is extremely simple and
trivial. Knowledge of using a text editor will suffice.

SciTE can syntax highlight code in various languages. It also has
auto-indentation, code-folding and other such features which are useful
when editing code.

SciTE also gives you the option to (compile and) run your code, from
within the editor.
\end{frame}



