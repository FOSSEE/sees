%%%%%%%%%%%%%%%%%%%%%%%%%%%%%%%%%%%%%%%%%%%%%%%%%%%%%%%%%%%%%%%%%%%%%%%%%%%%%%%%
% Using Linux Tools
%
% Author: FOSSEE 
% Copyright (c) 2009, FOSSEE, IIT Bombay
%%%%%%%%%%%%%%%%%%%%%%%%%%%%%%%%%%%%%%%%%%%%%%%%%%%%%%%%%%%%%%%%%%%%%%%%%%%%%%%%

\documentclass[12pt,compress]{beamer}

\mode<presentation>
{
  \usetheme{Warsaw}
  \useoutertheme{infolines}
  \setbeamercovered{transparent}
}

\usepackage[english]{babel}
\usepackage[latin1]{inputenc}
%\usepackage{times}
\usepackage[T1]{fontenc}

% Taken from Fernando's slides.
\usepackage{ae,aecompl}
\usepackage{mathpazo,courier,euler}
\usepackage[scaled=.95]{helvet}

\definecolor{darkgreen}{rgb}{0,0.5,0}

\usepackage{listings}
\lstset{language=sh,
    basicstyle=\ttfamily\bfseries,
    commentstyle=\color{red}\itshape,
  stringstyle=\color{darkgreen},
  showstringspaces=false,
  keywordstyle=\color{blue}\bfseries}

%%%%%%%%%%%%%%%%%%%%%%%%%%%%%%%%%%%%%%%%%%%%%%%%%%%%%%%%%%%%%%%%%%%%%%
% DOCUMENT STARTS
\begin{document}

\begin{frame}

\begin{center}
\vspace{12pt}
\textcolor{blue}{\huge Using Linux Tools\\Part II}
\end{center}
\vspace{18pt}
\begin{center}
\vspace{10pt}
\includegraphics[scale=0.95]{../images/fossee-logo.png}\\
\vspace{5pt}
\scriptsize Developed by FOSSEE Team, IIT-Bombay. \\ 
\scriptsize Funded by National Mission on Education through ICT\\
\scriptsize  MHRD,Govt. of India\\
\includegraphics[scale=0.30]{../images/iitb-logo.png}\\
\end{center}
\end{frame}
\begin{frame}
\frametitle{Objectives}
\label{sec-2}

At the end of this tutorial, you will be able to,
\begin{itemize}
\item Handle files efficiently.
\item Change permissions and ownership of files.
\item Navigate through directories and files.
\end{itemize}
\end{frame}

\begin{frame}
\frametitle{Pre-requisite}
\label{sec-3}

Spoken tutorial on -
\begin{itemize}
\item Using Linux tools -- Part I
\end{itemize}
\end{frame}


\begin{frame}
  \frametitle{Linux File Hierarchy}
  \begin{itemize}
  \item \texttt{/} is called the root directory
  \item It is the topmost level of the hierarchy
  \item For details \texttt{man hier}
  \end{itemize}
\end{frame}

\begin{frame}[fragile]
  \frametitle{Symbolic modes}
  \begin{small}
    \begin{center}
      \begin{tabular}{lll}
        Reference  &  Class   &  Description                                                      \\
        \hline
        u          &  user    &  the owner of the file                                            \\
        g          &  group   &  users who are members of the file's group                        \\
        o          &  others  &  users who are not hte owner of the file or members of the group  \\
        a          &  all     &  all three of the above; is the same as \emph{ugo}                \\
      \end{tabular}
    \end{center}

    \begin{center}
      \begin{tabular}{ll}
        Operator  &  Description                                                                   \\
        \hline
        +         &  adds the specified modes to the specified classes                             \\
        -         &  removes the specified modes from the specified classes                        \\
        =         &  the modes specified are to be made the exact modes for the specified classes  \\
      \end{tabular}
    \end{center}

    \begin{center}
      \begin{tabular}{lll}
        Mode  &  Name     &  Description                                 \\
        \hline
        r     &  read     &  read a file or list a directory's contents  \\
        w     &  write    &  write to a file or directory                \\
        x     &  execute  &  execute a file or recurse a directory tree  \\
      \end{tabular}
    \end{center}
  \end{small}
\end{frame}


\begin{frame}
  \frametitle{Exercise 1}
  \begin{itemize}
  \item Change the permissions of a directory along with all of its
        sub-directories and files.
  \end{itemize}
\end{frame}

\begin{frame}
  \frametitle{Solution 1}
  \begin{itemize}
  \item chmod go-r -R <directory name>/
  \end{itemize}
\end{frame}

\begin{frame}
\frametitle{Summary}
\label{sec-8}

  In this tutorial, we have learnt to,


\begin{itemize}
\item Copy and move files from one location to another, using the ``cp'' 
      and ``mv'' commands respectively.
\item Understand the Linux file hierarchy.
\item Change permissions and ownership of files, using the ``chmod''
      and ``chown'' commands respectively.
\end{itemize}
\end{frame}

\begin{frame}[fragile]
\frametitle{Evaluation}
\label{sec-9}


\begin{enumerate}
\item How to copy all the contents of one folder into another?
\vspace{15pt}
\item How will you rename the file wonderland.txt to alice.txt using the 
      commands learnt so far? 
\end{enumerate}
\end{frame}
\begin{frame}
\frametitle{Solutions}
\label{sec-10}


\begin{enumerate}
\item cp folder1/* folder2 
\vspace{15pt}
\item mv wonderland.txt alice.txt
\end{enumerate}
\end{frame}
\begin{frame}

  \begin{block}{}
  \begin{center}
  \textcolor{blue}{\Large THANK YOU!} 
  \end{center}
  \end{block}
\begin{block}{}
  \begin{center}
    For more Information, visit our website\\
    \url{http://fossee.in/}
  \end{center}  
  \end{block}
\end{frame}

\end{document}


