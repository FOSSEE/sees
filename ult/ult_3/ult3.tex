%%%%%%%%%%%%%%%%%%%%%%%%%%%%%%%%%%%%%%%%%%%%%%%%%%%%%%%%%%%%%%%%%%%%%%%%%%%%%%%%
% Using Linux Tools
%
% Author: FOSSEE 
% Copyright (c) 2009, FOSSEE, IIT Bombay
%%%%%%%%%%%%%%%%%%%%%%%%%%%%%%%%%%%%%%%%%%%%%%%%%%%%%%%%%%%%%%%%%%%%%%%%%%%%%%%%

\documentclass[12pt,compress]{beamer}

\mode<presentation>
{
  \usetheme{Warsaw}
  \useoutertheme{infolines}
  \setbeamercovered{transparent}
}

\usepackage[english]{babel}
\usepackage[latin1]{inputenc}
%\usepackage{times}
\usepackage[T1]{fontenc}

% Taken from Fernando's slides.
\usepackage{ae,aecompl}
\usepackage{mathpazo,courier,euler}
\usepackage[scaled=.95]{helvet}

\definecolor{darkgreen}{rgb}{0,0.5,0}

\usepackage{listings}
\lstset{language=sh,
    basicstyle=\ttfamily\bfseries,
    commentstyle=\color{red}\itshape,
  stringstyle=\color{darkgreen},
  showstringspaces=false,
  keywordstyle=\color{blue}\bfseries}

%%%%%%%%%%%%%%%%%%%%%%%%%%%%%%%%%%%%%%%%%%%%%%%%%%%%%%%%%%%%%%%%%%%%%%
% DOCUMENT STARTS
\begin{document}

\begin{frame}

\begin{center}
\vspace{12pt}

\textcolor{blue}{\huge Using Linux Tools\\Part III}
\end{center}
\vspace{18pt}
\begin{center}
\vspace{10pt}
\includegraphics[scale=0.95]{../images/fossee-logo.png}\\
\vspace{5pt}
\scriptsize Developed by FOSSEE Team, IIT-Bombay. \\ 
\scriptsize Funded by National Mission on Education through ICT\\
\scriptsize  MHRD,Govt. of India\\
\includegraphics[scale=0.30]{../images/iitb-logo.png}\\
\end{center}
\end{frame}
\begin{frame}
\frametitle{Objectives}
\label{sec-2}

At the end of this tutorial, you will be able to,
\begin{itemize}
\item Display the contents of files.
\item Read only parts of a file.
\item Look at the statistical information of a file.
\end{itemize}
\end{frame}

\begin{frame}
\frametitle{Pre-requisite}
\label{sec-3}

Spoken tutorial on -
\begin{itemize}
\item Using Linux tools -- Part I
\item Using Linux tools -- Part II
\end{itemize}
\end{frame}

\begin{frame}[fragile]
  \frametitle{\texttt{less}}
   \begin{itemize}
  \item q: Quit
  \item Arrows/Page Up/Page Down/Home/End: Navigation
  \item ng: Jump to line number n
  \item /pattern: Search. Regular expressions can be used
  \item h: Help
  \end{itemize}
\end{frame}

\begin{frame}
  \frametitle{Exercise 1}
  \begin{itemize}
  \item Print only the first, fifth and the seventh fields of the file ``/etc/passwd''.
  \end{itemize}
\end{frame}

\begin{frame}[fragile]
  \frametitle{\texttt{paste}}
      \begin{center}
      \begin{tabular}{l|l}
        \verb~students.txt~  &  \verb~marks.txt~  \\
        Hussain              &  89 92 85          \\
        Dilbert              &  98 47 67          \\
        Anne                 &  67 82 76          \\
        Raul                 &  78 97 60          \\
        Sven                 &  67 68 69          \\
      \end{tabular}
    \end{center}
\end{frame}

\begin{frame}
\frametitle{Summary}
\label{sec-8}

  In this tutorial, we have learnt to,


\begin{itemize}
\item Display the contents of files using the ``cat'' command.
\item View the contents of a file one screen at a time using the 
      ``less'' command.
\item Display specific contents of file using the ``head'' and 
      ``tail'' commands.
\item Use the ``cut'', ``paste'' and ``wc'' commands.
\end{itemize}
\end{frame}

\begin{frame}[fragile]
\frametitle{Evaluation}
\label{sec-9}


\begin{enumerate}
\item How to view lines from 1 to 15 in wonderland.txt ?
\vspace{15pt}
\item In ``cut'' command, how to specify space as the delimiter ? 
\end{enumerate}
\end{frame}
\begin{frame}
\frametitle{Solutions}
\label{sec-10}


\begin{enumerate}
\item \$ head -15 wonderland.txt
\vspace{15pt}
\item \$ cut -d " " <filename>
\end{enumerate}
\end{frame}
\begin{frame}

  \begin{block}{}
  \begin{center}
  \textcolor{blue}{\Large THANK YOU!} 
  \end{center}
  \end{block}
\begin{block}{}
  \begin{center}
    For more Information, visit our website\\
    \url{http://fossee.in/}
  \end{center}  
  \end{block}
\end{frame}

\end{document}


