%%%%%%%%%%%%%%%%%%%%%%%%%%%%%%%%%%%%%%%%%%%%%%%%%%%%%%%%%%%%%%%%%%%%%%%%%%%%%%%%
% Using Linux Tools
%
% Author: FOSSEE 
% Copyright (c) 2009, FOSSEE, IIT Bombay
%%%%%%%%%%%%%%%%%%%%%%%%%%%%%%%%%%%%%%%%%%%%%%%%%%%%%%%%%%%%%%%%%%%%%%%%%%%%%%%%

\documentclass[12pt,compress]{beamer}

\mode<presentation>
{
  \usetheme{Warsaw}
  \useoutertheme{infolines}
  \setbeamercovered{transparent}
}

\usepackage[english]{babel}
\usepackage[latin1]{inputenc}
%\usepackage{times}
\usepackage[T1]{fontenc}

% Taken from Fernando's slides.
\usepackage{ae,aecompl}
\usepackage{mathpazo,courier,euler}
\usepackage[scaled=.95]{helvet}

\definecolor{darkgreen}{rgb}{0,0.5,0}

\usepackage{listings}
\lstset{language=sh,
    basicstyle=\ttfamily\bfseries,
    commentstyle=\color{red}\itshape,
  stringstyle=\color{darkgreen},
  showstringspaces=false,
  keywordstyle=\color{blue}\bfseries}

%%%%%%%%%%%%%%%%%%%%%%%%%%%%%%%%%%%%%%%%%%%%%%%%%%%%%%%%%%%%%%%%%%%%%%
% DOCUMENT STARTS
\begin{document}

\begin{frame}

\begin{center}
\vspace{12pt}
\textcolor{blue}{\huge Using Linux Tools}
\end{center}
\vspace{18pt}
\begin{center}
\vspace{10pt}
\includegraphics[scale=0.95]{../images/fossee-logo.png}\\
\vspace{5pt}
\scriptsize Developed by FOSSEE Team, IIT-Bombay. \\ 
\scriptsize Funded by National Mission on Education through ICT\\
\scriptsize  MHRD,Govt. of India\\
\includegraphics[scale=0.30]{../images/iitb-logo.png}\\
\end{center}
\end{frame}
\begin{frame}
\frametitle{Objectives}
\label{sec-2}

At the end of this tutorial, you will be able to,
\begin{itemize}
\item Prepare a simple shell script. 
\item Run a script successfully and print it's result.
\item Understand what an environment variable is.
\end{itemize}
\end{frame}

\begin{frame}
\frametitle{Pre-requisite}
\label{sec-3}

Spoken tutorial on -
\begin{itemize}
\item Using Linux tools -- Part I
\item Using Linux tools -- Part II
\item Using Linux tools -- Part III
\item Using Linux tools -- Part IV
\item Using Linux tools -- Part V
\end{itemize}
\end{frame}

\begin{frame}[fragile]
  \frametitle{\texttt{PATH}}
  \begin{itemize}
  \item The shell searches in a set of locations, for the command 
  \item Locations are saved in ``environment'' variable called PATH
  \item \texttt{echo} can show the value of variables
  \end{itemize}
  \begin{lstlisting}
    $ echo $PATH
  \end{lstlisting} % $
  \begin{itemize}
  \item Put \texttt{results.sh} in one of these locations
  \item It can then be run without \texttt{./} 
  \end{itemize}
\end{frame}

\begin{frame}[fragile]
  \frametitle{Variables \& Comments}
  \begin{lstlisting}
    $ name=FOSSEE
    $ count=`wc -l wonderland.txt`
    $ echo $count # Shows the value of count
  \end{lstlisting} % $
  \begin{itemize}
  \item It is possible to create variables in shell scripts
  \item Variables can be assigned with the output of commands
  \item \alert{NOTE:} There is no space around the \texttt{=} sign
  \item All text following the \texttt{\#} is considered a comment
  \end{itemize}
\end{frame}

\begin{frame}
\frametitle{Summary}
\label{sec-8}

  In this tutorial, we have learnt to,


\begin{itemize}
\item Prepare a shell script.
\item Display the result of a script, using the ``echo'' command.
\item Use the environment variable ``PATH''.
\item Create variables and comment out content using the ``\#'' sign.
\end{itemize}
\end{frame}

\begin{frame}[fragile]
\frametitle{Evaluation}
\label{sec-9}


\begin{enumerate}
\item 
\item 
\item 
\end{enumerate}
\end{frame}
\begin{frame}
\frametitle{Solutions}
\label{sec-10}


\begin{enumerate}
\item 
\vspace{15pt}
\item 
\end{enumerate}
\end{frame}
\begin{frame}

  \begin{block}{}
  \begin{center}
  \textcolor{blue}{\Large THANK YOU!} 
  \end{center}
  \end{block}
\begin{block}{}
  \begin{center}
    For more Information, visit our website\\
    \url{http://fossee.in/}
  \end{center}  
  \end{block}
\end{frame}

\end{document}


