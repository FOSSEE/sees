%%%%%%%%%%%%%%%%%%%%%%%%%%%%%%%%%%%%%%%%%%%%%%%%%%%%%%%%%%%%%%%%%%%%%%%%%%%%%%%%
% Using Linux Tools
%
% Author: FOSSEE 
% Copyright (c) 2009, FOSSEE, IIT Bombay
%%%%%%%%%%%%%%%%%%%%%%%%%%%%%%%%%%%%%%%%%%%%%%%%%%%%%%%%%%%%%%%%%%%%%%%%%%%%%%%%

\documentclass[12pt,compress]{beamer}

\mode<presentation>
{
  \usetheme{Warsaw}
  \useoutertheme{infolines}
  \setbeamercovered{transparent}
}

\usepackage[english]{babel}
\usepackage[latin1]{inputenc}
%\usepackage{times}
\usepackage[T1]{fontenc}

% Taken from Fernando's slides.
\usepackage{ae,aecompl}
\usepackage{mathpazo,courier,euler}
\usepackage[scaled=.95]{helvet}

\definecolor{darkgreen}{rgb}{0,0.5,0}

\usepackage{listings}
\lstset{language=sh,
    basicstyle=\ttfamily\bfseries,
    commentstyle=\color{red}\itshape,
  stringstyle=\color{darkgreen},
  showstringspaces=false,
  keywordstyle=\color{blue}\bfseries}

%%%%%%%%%%%%%%%%%%%%%%%%%%%%%%%%%%%%%%%%%%%%%%%%%%%%%%%%%%%%%%%%%%%%%%
% DOCUMENT STARTS
\begin{document}

\begin{frame}

\begin{center}
\vspace{12pt}
\textcolor{blue}{\huge Using Linux Tools}
\end{center}
\vspace{18pt}
\begin{center}
\vspace{10pt}
\includegraphics[scale=0.95]{../images/fossee-logo.png}\\
\vspace{5pt}
\scriptsize Developed by FOSSEE Team, IIT-Bombay. \\ 
\scriptsize Funded by National Mission on Education through ICT\\
\scriptsize  MHRD,Govt. of India\\
\includegraphics[scale=0.30]{../images/iitb-logo.png}\\
\end{center}
\end{frame}
\begin{frame}
\frametitle{Objectives}
\label{sec-2}

At the end of this tutorial, you will be able to,
\begin{itemize}
\item Learn text editing tools of linux.
\end{itemize}
\end{frame}

\begin{frame}
\frametitle{Pre-requisite}
\label{sec-3}

Spoken tutorial on -
\begin{itemize}
\item Using Linux tools -- Part I
\item Using Linux tools -- Part II
\item Using Linux tools -- Part III
\item Using Linux tools -- Part IV
\end{itemize}
\end{frame}

\begin{frame}[fragile]
  \frametitle{\texttt{sort} \ldots}
  \begin{itemize}
  \item The command below sorts based on marks in first subject
  \end{itemize}
  \begin{lstlisting}
    $ cut -d " " -f 2- marks1.txt \
      | paste -d " " students.txt -\
      | sort -t " " -k 2 -rn
  \end{lstlisting} % $
  \begin{itemize}
  \item \texttt{-t} specifies the delimiter between fields
  \item \texttt{-k} specifies the field to use for sorting
  \item \texttt{-n} to choose numerical sorting
  \item \texttt{-r} for sorting in the reverse order
  \end{itemize}
\end{frame}

\begin{frame}[fragile]
  \frametitle{\texttt{tr}}
  \begin{itemize}
  \item \texttt{tr} translates or deletes characters
  \item Reads from \texttt{stdin} and outputs to \texttt{stdout}
  \item Given, two sets of characters, replaces one with other
  \item The following, replaces all lower-case with upper-case 
  \end{itemize}
  \begin{lstlisting}
    $ cat students.txt | tr a-z A-Z
  \end{lstlisting} % $
\end{frame}

\begin{frame}
\frametitle{Summary}
\label{sec-8}

  In this tutorial, we have learnt to,


\begin{itemize}
\item Use the ``sort'' command to sort lines of text files. 
\item Use the ``grep'' command to search text pattern.
\item Use the ``tr'' command to translate and/or delete characters.
\item Use the ``uniq'' command to omit repeated lines in a text. 
\end{itemize}
\end{frame}
\begin{frame}[fragile]
\frametitle{Evaluation}
\label{sec-9}


\begin{enumerate}
\item 
\item 
\item 
\end{enumerate}
\end{frame}
\begin{frame}
\frametitle{Solutions}
\label{sec-10}


\begin{enumerate}
\item 
\vspace{15pt}
\item 
\end{enumerate}
\end{frame}
\begin{frame}

  \begin{block}{}
  \begin{center}
  \textcolor{blue}{\Large THANK YOU!} 
  \end{center}
  \end{block}
\begin{block}{}
  \begin{center}
    For more Information, visit our website\\
    \url{http://fossee.in/}
  \end{center}  
  \end{block}
\end{frame}

\end{document}

