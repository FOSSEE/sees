%%%%%%%%%%%%%%%%%%%%%%%%%%%%%%%%%%%%%%%%%%%%%%%%%%%%%%%%%%%%%%%%%%%%%%%%%%%%%%%%
% Version Control Systems
%
% Author: FOSSEE 
% Copyright (c) 2009, FOSSEE, IIT Bombay
%%%%%%%%%%%%%%%%%%%%%%%%%%%%%%%%%%%%%%%%%%%%%%%%%%%%%%%%%%%%%%%%%%%%%%%%%%%%%%%%

\documentclass[14pt,compress]{beamer}

\mode<presentation>
{
  \usetheme{Warsaw}
  \useoutertheme{infolines}
  \setbeamercovered{transparent}
}

\usepackage[english]{babel}
\usepackage[latin1]{inputenc}
%\usepackage{times}
\usepackage[T1]{fontenc}

% Taken from Fernando's slides.
\usepackage{ae,aecompl}
\usepackage{mathpazo,courier,euler}
\usepackage[scaled=.95]{helvet}

\definecolor{darkgreen}{rgb}{0,0.5,0}

\usepackage{listings}
\lstset{language=Python,
    basicstyle=\ttfamily\bfseries,
    commentstyle=\color{red}\itshape,
  stringstyle=\color{darkgreen},
  showstringspaces=false,
  keywordstyle=\color{blue}\bfseries}

%%%%%%%%%%%%%%%%%%%%%%%%%%%%%%%%%%%%%%%%%%%%%%%%%%%%%%%%%%%%%%%%%%%%%%
% Macros
\setbeamercolor{emphbar}{bg=blue!20, fg=black}
\newcommand{\emphbar}[1]
{\begin{beamercolorbox}[rounded=true]{emphbar} 
      {#1}
 \end{beamercolorbox}
}
\newcounter{time}
\setcounter{time}{0}
\newcommand{\inctime}[1]{\addtocounter{time}{#1}{\tiny \thetime\ m}}

\newcommand{\typ}[1]{\lstinline{#1}}

\newcommand{\kwrd}[1]{ \texttt{\textbf{\color{blue}{#1}}}  }

%%% This is from Fernando's setup.
% \usepackage{color}
% \definecolor{orange}{cmyk}{0,0.4,0.8,0.2}
% % Use and configure listings package for nicely formatted code
% \usepackage{listings}
% \lstset{
%    language=Python,
%    basicstyle=\small\ttfamily,
%    commentstyle=\ttfamily\color{blue},
%    stringstyle=\ttfamily\color{orange},
%    showstringspaces=false,
%    breaklines=true,
%    postbreak = \space\dots
% }

%%%%%%%%%%%%%%%%%%%%%%%%%%%%%%%%%%%%%%%%%%%%%%%%%%%%%%%%%%%%%%%%%%%%%%
% Title page
\title[Version Control Systems]{SEES: Version Control Systems}

\author[FOSSEE] {FOSSEE}

\institute[IIT Bombay] {Department of Aerospace Engineering\\IIT Bombay}
\date[]{}
%%%%%%%%%%%%%%%%%%%%%%%%%%%%%%%%%%%%%%%%%%%%%%%%%%%%%%%%%%%%%%%%%%%%%%

%\pgfdeclareimage[height=0.75cm]{iitmlogo}{iitmlogo}
%\logo{\pgfuseimage{iitmlogo}}


%% Delete this, if you do not want the table of contents to pop up at
%% the beginning of each subsection:
\AtBeginSubsection[]
{
  \begin{frame}<beamer>
    \frametitle{Outline}
    \tableofcontents[currentsection,currentsubsection]
  \end{frame}
}

\AtBeginSection[]
{
  \begin{frame}<beamer>
    \frametitle{Outline}
    \tableofcontents[currentsection,currentsubsection]
  \end{frame}
}

% If you wish to uncover everything in a step-wise fashion, uncomment
% the following command: 
%\beamerdefaultoverlayspecification{<+->}

%%\includeonlyframes{current,current1,current2,current3,current4,current5,current6}

%%%%%%%%%%%%%%%%%%%%%%%%%%%%%%%%%%%%%%%%%%%%%%%%%%%%%%%%%%%%%%%%%%%%%%
% DOCUMENT STARTS
\begin{document}

\begin{frame}
  \maketitle
\end{frame}

% CREATING TOC 
\begin{frame}
  \frametitle{Outline}
  \tableofcontents
  % You might wish to add the option [pausesections]
\end{frame}

%% There are some %$ used just to minimise the effect of $ sign used in lstlisting. In emacs it looks unhealthy.

% Introduction to course-need of version control, history, options available.
\section{Introduction}

\begin{frame}
  \frametitle{What is Version Control?}
  \begin{block}{From a blog post}
    ``Version control (or source control) is nothing more arcane than keeping copies of ones work as one make changes to it.''
  \end{block}
  \pause
  \begin{block}{}
    It is better to use these tools rather then wasting creativity to invent VCS which have files with names like \begin{color}{red}{prog1.py, prog2.py}\end{color} or \begin{color}{red}ass1.py, ass2.py.\end{color}
  \end{block}
\end{frame}

\begin{frame}
  \frametitle{Motivation behind such tools}
  \begin{itemize}
  \item Track the history and evolution of a program.
  \item To collaborate effectively on a project.
  \item \begin{color}{red}``To err is Human''\end{color} \pause for recovery we have ``Version Control''
  \end{itemize}
\end{frame}

\begin{frame}
  \frametitle{How is done What is done?}
  \begin{itemize}
  \item It keeps track of changes you make to a file. You can improvise, revisit, and amend.
  \item all procedure is logged/recorded, so you and others can follow the development cycle.
  \end{itemize}  
\end{frame}

\begin{frame}
  \frametitle{Do we really need this?}
  \emphbar{For team of people working remotely(even different computers/machines) on a project use of version control is inevitable!}
  \emphbar{For single person: managing projects and assignments becomes easy}
  \pause
  \emphbar{Its a good habit!}
\end{frame}

\begin{frame}
  \frametitle{Whats on the menu!}
  \begin{itemize}
  \item cvs(Concurrent Version System)
  \item svn (Subversion)
  \item hg (Mercurial)
  \item bzr (Bazaar)
  \item git
  \end{itemize}
\end{frame}

% Introduction to jargons 
\section{Learning the Lingo!}

\begin{frame}
  \frametitle{Common jargons: Basic setup}
  \begin{itemize}
  \item Repository(repo):\\
        The folder with all files.
  \item Server:\\
        Machine with main inventory/repo.
  \item Client:\\
        Local machines with copy of main repo.
  \end{itemize}
\end{frame}

\begin{frame}
  \frametitle{Actions}
  \begin{itemize}
  \item Add:\\
    Adding file into the repo for the first time.
  \item Version:\\
    Version number of a file.
  \item Head/Tip:\\
    The latest revision of the repo.
  \item Check out/Clone:\\
    Initial download of repo onto machine.
  \item Commit:\\
    Recording a change.
  \item Change log/History:\\
    List of changes made to repo.
  \end{itemize}
\end{frame}

\begin{frame}
  \frametitle{Actions cont...}
  \begin{itemize}
  \item Branch:\\
    Separate local copy of repo for bug fixing, testing.
  \item Diff/Change:\\
    Finding the differences in a file in two versions.
  \item Merge (or patch):\\
    Apply the changes to file, to make it up-to-date.
  \item Conflict:\\
    When merging a file creates a contradict.
  \item Resolve:\\
    Fixing the conflict manually.
  \end{itemize}
\end{frame}

% Types of Version Controls
\section{Types of VCS}

\begin{frame}
  \frametitle{Types:}
  Based on way of managing the repo there are two types of VCS:
  \begin{itemize}
  \item Centralized VCS
  \item Distributed VCS
  \end{itemize}
\end{frame}

\begin{frame}
  \frametitle{Why hg?}
  \begin{itemize}
  \item Easy to learn and use.
  \item Lightweight.
  \item Scales excellently.
  \item Based on Python.
  \end{itemize}
\end{frame}

% Initializing the repo, cloning, committing changes, pushing, pulling to repo.
\section{Getting Started}

\begin{frame}[fragile]
  \frametitle{Getting comfortable:}
  Try following commands:
  \begin{lstlisting}
    $ hg version    
    $ hg help
    $ hg help diff
  \end{lstlisting} %$
\end{frame}

\begin{frame}[fragile]
  \frametitle{Cloning a repo}
  Clone is used to make a copy of an existing repository. This repo can be both local or remote.
  \begin{lstlisting}
$ hg clone http://hg.serpentine.com/tutorial/hello 
     localCopyhello
  \end{lstlisting}
  And we get a local copy of this repository. 
  \begin{lstlisting}
$ ls localCopyhello/
hello.c  Makefile
  \end{lstlisting}
\end{frame}

\begin{frame}[fragile]
  \frametitle{Creating a repo}
  To start a new repository \typ{hg} provides \typ{init} command.
  \begin{lstlisting}
$ mkdir Fevicol
$ cd Fevicol/
$ echo "print 'Yeh Fevicol ka majboot 
              jod hai'" > feviStick.py
$ ls -a
.  ..  feviStick.py
$ hg init
$ ls -a
.  ..  feviStick.py  .hg
  \end{lstlisting}
\typ{.hg} folder will store the logs related to this repo.
\end{frame}

\begin{frame}[fragile]
  \frametitle{Cloning a local repo: Branching}
  All \typ{hg} repositories are self-contained, and independent which can be cloned. Like:
  \begin{lstlisting}
$ hg clone localCopyhello newCopy
updating working directory
2 files updated, 0 files merged, 
0 files removed, 0 files unresolved
  \end{lstlisting}
or
  \begin{lstlisting}
$ hg clone Fevicol Fevicol-pull
updating working directory
0 files updated, 0 files merged, 
0 files removed, 0 files unresolved
  \end{lstlisting}
\end{frame}

\begin{frame}[fragile]
  \frametitle{History/Logs}
  To check out track record of a repo one has to use \typ{log} command.
  \begin{lstlisting}
$ cd localCopyhello
$ hg log    
  \end{lstlisting}
  The output of previous command have following fields:
  \begin{itemize}
  \item changeset: identifiers for the changeset.
  \item user: person who creates the changeset.
  \item date: The date and time of creation of changeset.
  \item summary: The one line description of changeset.
  \end{itemize}
\end{frame}

\begin{frame}[fragile]
  \frametitle{History/Logs cont...}
By default it returns complete logs of all changes. To make it selective try:
\begin{lstlisting}
  $ hg log -r 3
  $ hg log -r 2:4
\end{lstlisting}
To see tip/latest commit history use:\\
\typ{$ hg tip} %$
\end{frame}

\begin{frame}[fragile]
  \frametitle{Adding files}
  We will revisit the Fevicol repo we created earlier.
  \begin{lstlisting}
$ cd Fevicol
$ hg log
$ hg status
? feviStick.py
  \end{lstlisting} %$
  "?" sign in front of name indicates that this file is not yet part of track record. \typ{add} command is used to add new files to repo.
  \begin{lstlisting}
$ hg add feviStick.py
$ hg st
A feviStick.py
  \end{lstlisting}
\end{frame}

\begin{frame}[fragile]
  \frametitle{Committing changes}
  \typ{hg} uses \typ{commit} (alias \typ{ci}) command to make changes logged. So after adding a file, we have to commit it also by:
  \begin{lstlisting}
$ hg ci -u "Shantanu <shantanu@fossee.in>" 
        -m "First commit."
$ hg log
changeset:   0:84f5e91f4de1
tag:         tip
user:        Shantanu <shantanu@fossee.in>
date:        Fri Aug 21 23:37:13 2009 +0530
summary:     First commit.    
  \end{lstlisting}
\end{frame}

\begin{frame}[fragile]
  \frametitle{More basic operations}
  \begin{lstlisting}
$ hg rename feviStick.py feviCol.py
$ hg st
A feviCol.py
$ hg ci -u "Shantanu <shantanu@fossee.in>" 
        -m "Renamed feviStick.py."
$ hg tip
changeset:   1:d948fb4137c5
tag:         tip
user:        Shantanu <shantanu@fossee.in>
date:        Sat Aug 22 00:11:25 2009 +0530
summary:     Renamed feviStick.py.
  \end{lstlisting}
%% Other commands which can be handy are \typ{cp}, \typ{remove}, \typ{revert} etc.
\end{frame}

% Introduction to concepts of branches, merging patch?
\section{Sharing and Collaborating}

\begin{frame}[fragile]
  \frametitle{Distributing changes}
  As this repo is self-contained, hence changeset just created are local and are not propagated to previously cloned Fevicol-pull.
  \begin{lstlisting}
$ hg pull 
pulling from /home/baali/Fevicol
requesting all changes
adding changesets
adding manifests
adding file changes
added 2 changesets with 2 changes to 2 files
(run 'hg update' to get a working copy)
  \end{lstlisting} %$
\end{frame}

\begin{frame}[fragile]
  \frametitle{Pulling changesets cont...}
  as last line of previous command suggest, hg \typ{pull} does not(by default) update current directory. It just imports changesets. To add all these changesets one have to update using \typ{up} command:
  \begin{lstlisting}
$ cd Fevicol-pull
$ ls -a
.  ..  .hg
$ hg up
2 files updated, 0 files merged, 
0 files removed, 0 files unresolved
$ ls -a
.  ..  feviCol.py  feviStick.py  .hg    
  \end{lstlisting}
\end{frame}

\begin{frame}[fragile]
  \frametitle{Making changes across the repos}
  \typ{$ cd Fevicol-clone/}\\ %$
  Lets edit and correct the feviStick.py 
\begin{lstlisting}
$ echo "print 'Ab no more Chip Chip'" 
        > feviStick.py
$ hg st
M feviStick.py
\end{lstlisting}
  'M' sign indicates that Mercurial has noticed change.\\
\end{frame}

\begin{frame}[fragile]
  \frametitle{Revisiting changes}
To look back at the changes made there is \typ{diff} command:
\begin{lstlisting}
$ hg diff
diff -r a7912d45f47c feviStick.py
--- a/feviStick.py   Sun Aug 23 22:34:35 2009 +0530
+++ b/feviStick.py   Sun Aug 23 22:47:49 2009 +0530
@@ -1,1 +1,1 @@
-print 'Yeh Fevicol ka Majboot jod hai'
+print 'Ab no more Chip Chip'
  \end{lstlisting} %$
  These changes are not logged until you commit them.\\
  \begin{lstlisting}
$ hg ci -u "Shantanu <shantanu@fossee.in>" 
      -m "Changed tagline for feviStick.py."
  \end{lstlisting} %$
\end{frame}

\begin{frame}[fragile]
  \frametitle{Syncing two repos}
  Now to bring both the repos to same stage one have to \typ{push} changes.
  \begin{lstlisting}
$ hg push 
pushing to /home/baali/Fevicol
searching for changes
adding changesets
adding manifests
adding file changes
added 1 changesets with 1 changes to 1 files
  \end{lstlisting} %$
\end{frame}


\begin{frame}[fragile]
  \frametitle{Syncing cont...}
  Same as pulling, pushing wont update the main repo by default. Try running following command:
  \begin{lstlisting}
$ hg tip    
$ cat feviStick.py
  \end{lstlisting}
  \typ{tip} shows latest changeset, but content of file are not updated. We have to use \typ{up} on main branch
  \begin{lstlisting}
$ hg up
1 files updated, 0 files merged, 0 files removed, 0 files unresolved    
  \end{lstlisting} %$
\end{frame}

\begin{frame}[fragile]
  \frametitle{Merging: Scenario}
  One very useful feature is merging work of different peers working on same project.\\
  We consider scenario, two person on one project, both have local copies, and one among them is main branch.
\end{frame}


\begin{frame}[fragile]
  \frametitle{Making changes to one of repo}
  \begin{lstlisting}
$ cd Fevicol-pull
$ echo "print 'Yeh Fevicol ka Majboot jod 
        hai, tootega nahin'" > feviCol.py
$ hg st
M feviStick.py
$ hg ci -u "Shantanu <shantanu@fossee.in>" 
     -m "Updated tag line for feviCol.py."
$ hg tip| grep changeset
changeset:   4:caf986b15e05
  \end{lstlisting} %$
\end{frame}

\begin{frame}[fragile]
  \frametitle{In the meanwhile, other repo is ...}
  \begin{lstlisting}
$ cd Fevicol
$ echo "print 'Jor laga ke hayyiya'" 
        > firstAdd.py
$ hg add 
$ hg st
A firstAdd.py
$ hg ci -u "Shantanu <shantanu@fossee.in>"
        -m "Added firsAdd.py."
$ hg tip|grep changeset
changeset:   4:fadbd6492cc4    
  \end{lstlisting}
\end{frame}

\begin{frame}[fragile]
  \frametitle{Merging}
  \begin{lstlisting}
$ hg pull ../Fevicol-pull
pulling from ../Fevicol-pull
searching for changes
adding changesets
adding manifests
adding file changes
added 1 changesets with 1 changes to 1 files (+1 heads)
(run 'hg heads' to see heads, 'hg merge' to merge)    
  \end{lstlisting} %$
  Output is already suggesting something!
\end{frame}


\begin{frame}[fragile]
  \frametitle{Analyzing events in detail}
  Since hg \typ{pull} don't update the files directly, our changes are still safe. Following command can help us deal this merging problem in better way:
  \begin{lstlisting}
$ hg heads
  \end{lstlisting}
  This commands shows repo/branch heads.
  \begin{lstlisting}
$ hg glog    
  \end{lstlisting}
  It shows revision history alongside an ASCII revision graph.\\
  Because of different track, \typ{up} command fails.
  \begin{lstlisting}
$ hg up
abort: crosses branches (use 'hg merge' or 'hg update -C')
  \end{lstlisting} %$
\end{frame}

\begin{frame}[fragile]
  \frametitle{Merging}
  \typ{hg merge} command merge working directory with another revision.
  \begin{lstlisting}
$ hg merge    
  \end{lstlisting} %$
  After merging two branches, we have to commit the results to create a common head.
  \begin{lstlisting}
$ hg ci -u "Shantanu <shantanu@fossee.in>" 
        -m "Merged branches."
$ hg heads    
$ hg glog
  \end{lstlisting} %$
\end{frame}

% Steps to follow to make life easier. How to avoid/handle manual merges.
\section{Work flow: DOS and DON'Ts}

\begin{frame}
  \frametitle{Motto behind hg}
  \begin{center}
  \color{red}{``Commit Early Commit Often.''}\\  
  \end{center}  
\end{frame}

\begin{frame}
  \frametitle{Steps to be followed}
  \begin{itemize}
  \item Make changes.
  \item Commit.
  \item Pull changesets.
  \item Merge if required.
  \item Push.
  \end{itemize}
\end{frame}

\begin{frame}
  \frametitle{Suggested Readings:}
  \begin{itemize}
  \item \url{http://hgbook.red-bean.com/}
  \item \url{http://karlagius.com/2009/01/09/version-control-for-the-masses/}
  \item Articles related to version control available on \url{http://betterexplained.com/}
  \item \url{http://en.wikipedia.org/wiki/Revision_control}
  \item \url{http://wiki.alliedmods.net/Mercurial_Tutorial}
  \end{itemize}
\end{frame}
\end{document}
