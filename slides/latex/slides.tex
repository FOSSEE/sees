\documentclass{beamer}
\usepackage[latin1]{inputenc}
\usepackage[T1]{fontenc}
\usepackage{fixltx2e}
\usepackage{graphicx}
\usepackage{longtable}
\usepackage{float}
\usepackage{wrapfig}
\usepackage{soul}
\usepackage{textcomp}
\usepackage{marvosym}
\usepackage{wasysym}
\usepackage{latexsym}
\usepackage{amssymb}
\usepackage{hyperref}
\tolerance=1000
\usepackage[english]{babel} \usepackage{ae,aecompl}
\usepackage{mathpazo,courier,euler} \usepackage[scaled=.95]{helvet}
\usepackage{listings}
\lstset{
  language=TeX,
  basicstyle=\ttfamily\bfseries,
  commentstyle=\ttfamily\color{blue},
  stringstyle=\ttfamily\color{orange},
  showstringspaces=false,
  breaklines=true,
  postbreak = \space\dots
}

\newcommand{\typ}[1]{\lstinline{#1}}

\mode<presentation>
{
  \usetheme{Warsaw}
  \useoutertheme{infolines}
  \setbeamercovered{transparent}
}


\title{\LaTeX}
\author[FOSSEE] {FOSSEE}
\institute[IIT Bombay] {Department of Aerospace Engineering\\IIT
  Bombay}
\date{}

%% Delete this, if you do not want the table of contents to pop up at
%% the beginning of each subsection:
\AtBeginSubsection[]
{
  \begin{frame}<beamer>
    \frametitle{Outline}
    \tableofcontents[currentsection,currentsubsection]
  \end{frame}
}

\AtBeginSection[]
{
  \begin{frame}<beamer>
    \frametitle{Outline}
    \tableofcontents[currentsection,currentsubsection]
  \end{frame}
}

\begin{document}

% Document title
\begin{frame}
  \maketitle  
\end{frame}

\section{Introduction}

\begin{frame}
  \frametitle{{\LaTeX} - Introduction}
  \begin{itemize}
  \item Typesetting program
    \begin{itemize}
    \item What is typesetting?
    \end{itemize}
  \item Excellently Typeset Documents - specially Math
  \item Anything from one page articles to huge books
  \item Pronounced \emph{Lah-tech} or \emph{Lay-tech}
  \end{itemize}
\end{frame}

\begin{frame}
  \frametitle{Why \LaTeX?}
  \begin{itemize}
  \item Excellent visual quality! 
  \item Handles the typesetting; Lets you focus on content
  \item Makes writing math extremely simple
  \item It is a standard -- widely used in Scientific community
  \end{itemize}
  \begin{block}{}
    \[\tilde{N}_{\mathbf{x}}\times \mathbf{r}(\mathbf{x}) f_{1k}(\mathbf{x},t) - \frac{1}{2} \tilde{N} \tilde{N}:\mathbf{BB}^{T}P(\mathbf{x},t) = -m_{k}f_{1k}(\mathbf{x},t) + 2 \mathop{\mathbf{\aa}}_{j=1}^{K} f_{1j}(\mathbf{x},t)m_{j}P_{k|j} \]
  \end{block}
\end{frame}

\begin{frame}
  \frametitle{Course Outline}
  \begin{itemize}
  \item Look at the sample document - \typ{sample.pdf}
    \begin{itemize}
    \item Title, Author, Date
    \item Abstract
    \item Sections \& Subsections
    \item Appendix
    \item References/Bibliography
    \item Tables
    \item Figures
    \item Math
    \end{itemize}
  \item The document will be produced by the end of the course. 
  \item First Hour - Basic Structure
  \item Second Hour - Text, Tables, Figures, References
  \item Third Hour - Math, Bibliography, Presentations
  \end{itemize}
\end{frame}

\begin{frame}[fragile]
  \frametitle{\LaTeX~ as a Mark-up}
  \begin{itemize}
  \item {\LaTeX} is a document based mark-up
  \item Mark-up --- a system of annotating text, adding extra
    information to specify structure and presentation of text
  \item Document based markup $\rightarrow$ you don't have to worry
    about each element individually 
  \item Allows you to focus on content, rather than appearance.
  \end{itemize}
\end{frame}

\begin{frame}[fragile]
  \frametitle{Typesetting a minimal document}
  \begin{itemize}
  \item Write the sample code  into the file \typ{draft.tex}\\
      {\tiny See \typ{hg} rev0 of draft}
  \item   To compile, (in terminal) \\
    \begin{lstlisting}[language=bash]
      $ pdflatex draft.tex
    \end{lstlisting} %%$
  \item This produces the output file \typ{draft.pdf} 
  \item \alert{Note:} \typ{latex} vs. \typ{pdflatex} 
  \end{itemize}
\end{frame}  

\begin{frame}[fragile]
  \frametitle{Commands \& Environments}
  \begin{itemize}
  \item {\LaTeX} is case sensitive
  \item Commands begin with a \typ{\\}
  \item Environments have a \typ{\\begin} and \typ{\\end} 
  \item Any content after the \typ{\\end\{document\}} is ignored
  \end{itemize}
\end{frame}  

\begin{frame}[fragile]
  \frametitle{Comments \& Special Characters}
  \begin{itemize}
  \item Anything that follows a \typ{\%} symbol till end of the line
    is a comment 
  \item Special characters (\typ{\~ \# \$ \^ \& \_ \{ \}}) are escaped by a
    \typ{\\} 
  \item \typ{\\} symbol is inserted using \typ{\\textbackslash}
    command
  \end{itemize}
\end{frame}  

\begin{frame}[fragile]
  \frametitle{Spacing}
  \begin{itemize}
  \item \typ{\\\\} inserts a new line in the output
  \item An empty line marks the beginning of a new paragraph
  \item Multiple spaces (or empty lines) are equivalent to a single
    space (or empty line)
  \end{itemize}
\end{frame}  

\section{Adding Structure}

\begin{frame}[fragile]
  \frametitle{\typ{documentclass}}
  \begin{itemize}
  \item Used to select the \emph{class} of our document
  \item Some available classes - \typ{article}, \typ{proc},
    \typ{report}, \typ{book}, \typ{slides}, \typ{letter}.
   \item For example:
    \typ{\\documentclass\[12pt,a4paper,draft\]\{report\}}\\
    The parameters within \typ{\[ \]} are optional.
    \begin{itemize}
    \item \typ{12pt} -- sets the font size of main font and others are
      relatively, adjusted. \typ{10pt} is the default. 
    \item \typ{a4paper} -- specify paper size
    \item \typ{draft} -- marks hyphenation and justification problems in
      typesetting with a square in the margin
    \end{itemize}

  \end{itemize}
\end{frame}

\begin{frame}[fragile]
  \frametitle{Top Matter}
  Let's add the title, author's name and the date.
  \begin{itemize}
  \item Add title, author and date. 
  \item Compile. 
  \item Nothing changes.
  \end{itemize}
  {\tiny See \typ{hg} rev1 of draft.}
\end{frame}

\begin{frame}[fragile]
  \frametitle{Top Matter \ldots}
  \begin{itemize}
  \item \lstinline{\maketitle} command inserts the top-matter.
  \item Add the command to the document \& compile again. 
  \item If no date is specified, today's date is automatically
    inserted.
  \end{itemize}
  \tiny{See \typ{hg} rev2 of draft.}
\end{frame}


\begin{frame}[fragile]
  \frametitle{Abstract}
  \begin{itemize}
  \item \typ{abstract} environment inserts abstract.
  \item Place it at the location where you want your abstract. 
  \end{itemize}
  \tiny See rev3 of \typ{hg}
\end{frame}

\begin{frame}[fragile]
  \frametitle{Sectioning}
  \begin{itemize}
  \item \lstinline{\section}, \lstinline{\subsection}
    \lstinline{\subsubsection}
  \item Auto numbered sections!
  \item \typ{*} to prevent numbering of a section
  \end{itemize}
  \tiny See rev4 of \typ{hg}
\end{frame}

\begin{frame}[fragile]
  \frametitle{Sectioning \ldots}
  \begin{itemize}
  \item Longer documents, use \lstinline{report} or \lstinline{book}
    class
  \item Chapter can be added using \lstinline{\chapter}
  \end{itemize}
  \begin{lstlisting}
    \documentclass{report}

    \chapter{One}
  \end{lstlisting}
  \begin{itemize}
  \item subsections do not get numbering
  \item Change \lstinline{secnumdepth}
  \end{itemize}
  \begin{lstlisting}
    \setcounter{secnumdepth}{3}
  \end{lstlisting}
   \tiny See rev5 of \typ{hg}
\end{frame}

\begin{frame}[fragile]
  \frametitle{Appendices}
  \begin{itemize}
  \item \lstinline{\appendix} command indicates the beginning of
    Appendices. 
  \item Any content after \lstinline{\appendix}, will be added to the
    appendix 
  \item Use sectioning commands to add sections
  \end{itemize}
  \tiny See rev7 of \typ{hg}
\end{frame}

\begin{frame}[fragile]
  \frametitle{Table of Contents [TOC]}
  \begin{itemize}
  \item Our document is short, but let's learn to add a TOC
  \item Add \lstinline{\tableofcontents} where you want TOC to
    appear
  \item Compile
  \item Only headings appear. No page numbers
  \item A \lstinline{.toc} file is generated
  \item Re-compile
  \item Any numbered section/block automatically appears
  \end{itemize}
  \tiny See rev8 of \typ{hg}
\end{frame}

\begin{frame}[fragile]
  \frametitle{TOC \ldots}
  \begin{itemize}
  \item Un-numbered sections are added to TOC using
    \lstinline{\addcontentsline}
  \item For instance,  \lstinline+\addcontentsline{toc}{section}{Intro}+
  \end{itemize}
  \tiny See rev9 of \typ{hg}
\end{frame}

\begin{frame}
  \frametitle{Bibliography}
  We shall look at Bibliographies, later in the course. 
\end{frame}

\section{Typesetting Text}
\begin{frame}[fragile]
  \frametitle{Quotation Marks}
  \begin{itemize}
  \item Use \`~ (accent) for left quote
  \item Use \'~ (apostrophe) for right quote
  \item For double quotes, use them twice
  \end{itemize}
  \tiny See rev11 of \typ{hg}
\end{frame}

\begin{frame}[fragile]
  \frametitle{Fonts - Emphasis, Fixed width, \ldots}
  \begin{itemize}
  \item \lstinline{\emph} gives emphasized or italic text
  \item \typ{flushleft} to have text left aligned
  \item \typ{flushright}, \typ{center}
  \end{itemize}
  \tiny See rev12 of \typ{hg}
\end{frame}

\begin{frame}[fragile]
  \frametitle{Fonts - Emphasis, Fixed width, \ldots}
  \begin{itemize}
  \item \lstinline{\texttt} gives fixed width font
  \item \lstinline{\textbf} bold face font
  \item \lstinline{--} en dash (--); \lstinline{---} em dash (---). 
  \end{itemize}
  \tiny See rev13 of \typ{hg}
\end{frame}

\begin{frame}[fragile]
  \frametitle{Lists}
  \begin{itemize}
  \item \lstinline{enumerate} environment is used for numbered lists
  \item \lstinline{itemize} environment gives un-numbered lists
  \item Each item in the list is specified using \lstinline{\item}
  \item Nested lists are also easily handled, as expected
  \end{itemize}
  \tiny See rev14 of \typ{hg}
\end{frame}

\begin{frame}[fragile]
  \frametitle{Footnotes}
  \begin{itemize}
  \item \typ{\\footnote} command adds a footnote
  \end{itemize}
  \tiny See rev15 of \typ{hg}
\end{frame}

\begin{frame}[fragile]
  \frametitle{Labels and References}
  \begin{itemize}
  \item \lstinline+\label{labelname}+ is used to label an element
  \item \lstinline+\ref{labelname}+ is used to refer to that element
  \item Compile twice
  \end{itemize}
  \tiny See rev15 of \typ{hg}
\end{frame}

\begin{frame}[fragile]
  \frametitle{Including code}
  \begin{itemize}
  \item Instead of using \lstinline{\texttt} we could use
    \lstinline{\verbatim} 
  \item \lstinline+listings+ is a powerful package
  \item \lstinline+\usepackage{listings}+ needs to be added 
  \item Tell {\LaTeX} the language to be used, using \typ{\\lstset}
  \end{itemize}
  \tiny See rev16 of \typ{hg}
\end{frame}

\begin{frame}[fragile]
  \frametitle{Including code}
  \begin{itemize}
  \item Use \lstinline+\lstlisting+ for a block of code
  \item \typ{\\lstinline} for inline code
  \end{itemize}
  \tiny See rev16 of \typ{hg}
\end{frame}

\section{Figures, Tables \& Floats}
\begin{frame}[fragile]
  \frametitle{Figures}
  \begin{itemize}
  \item The \typ{graphicx} package allows us to insert graphics
  \item \lstinline+\usepackage{graphicx}+
  \item To add a graphic, use \lstinline{\includegraphics} command
  \item Use relative path to the image
  \end{itemize}
  \tiny See rev17 of \typ{hg}
\end{frame}

\begin{frame}[fragile]
  \frametitle{\lstinline{includgraphics}}
  It takes following optional arguments
  \begin{itemize}
  \item \lstinline+scale+ --- specifies the factor by which to scale
    the image 
  \item \lstinline+height+, \lstinline+width+ --- If only one of them
    is specified, aspect ratio is maintained 
  \item \lstinline+keepaspectratio+ --- boolean value to keep aspect
    ratio or not 
  \item \lstinline+angle+ --- specify by what angle the image should
    be rotated 
  \end{itemize}
\end{frame}

\begin{frame}[fragile]
  \frametitle{Floats}
  \begin{itemize}
  \item Graphics (\& Tables) are special because they cannot be broken
    across pages 
  \item They are ``floated'' to the next page, if they don't fit in
    the current page 
  \item Enclose graphic within \lstinline+figure+ environment to make
    it float 
  \item Figure environment takes additional parameter for location of
    float 
  \end{itemize}
  \begin{table}
    \caption{Permission Specifiers}
    
    \begin{tabular}{|c|c|}
      Specifier & Permission\\\hline
      t & Top of page\\
      b & Bottom of page\\
      p & Separate page for floats\\
      h & here (the same place where command appears in source)\\
      ! & override \LaTeX's internal parameters for good position
    \end{tabular}
  \end{table}
\end{frame}

\begin{frame}
  \frametitle{Captions and References}
  \begin{itemize}
  \item Figure environment allows us add a caption
  \item To place the image in the center we enclose it in the
    \lstinline+center+ environment 
  \item We can label images too
  \item label should be added after the caption command
  \item Figures are auto numbered
  \end{itemize}
  \tiny See rev17 of \typ{hg}
\end{frame}

\begin{frame}[frame]
  \frametitle{Tables}
  \begin{itemize}
  \item \lstinline+tabular+ is used to typeset a table
  \item It is enclosed in a \lstinline+table+ environment to make it a
    float 
  \item \lstinline+table+ environment also gives captions, auto
    numbering  
  \end{itemize}
\end{frame}


\begin{frame}[fragile]
  \frametitle{\lstinline+tabular+}
  \begin{itemize}
  \item tabular takes formatting of each column as argument
  \end{itemize}

  \begin{table}
    \caption{tabular environment}
    
    \begin{tabular}{|l|l|}
      \lstinline+l+ & left justified column content\\\hline
      \lstinline+r+ & right justified column content\\\hline
      \lstinline+c+ & centered column content\\\hline
      \lstinline+|+ & produces a vertical line\\
    \end{tabular}
  \end{table}
  \begin{itemize}
  \item also takes an optional parameter for specifying position of
    table 
  \item \lstinline+t+ for top, \lstinline+b+ for bottom, \lstinline+c+
    for center 
  \item each column of table is separated by \&
  \item each row is separated by newline \lstinline{\\}
  \item \lstinline+\hline+ give a horizontal line between two rows
  \end{itemize}
  \tiny See rev18 of \typ{hg}
\end{frame}

\begin{frame}[fragile]
  \frametitle{List of Tables, Figures}
  \begin{itemize}
  \item \lstinline+\listoftables+ -- to add a list of tables
  \item \lstinline+\listoffigures+ -- to add a list of figures
  \end{itemize}
\end{frame}


\section{Typesetting Math}
\begin{frame}[fragile]
  \frametitle{Math in \LaTeX}
  \begin{itemize}
  \item Math is enclosed in a pair of \lstinline{$} signs or %%$
    \lstinline+\(  \)+ 
  \item Used for typesetting inline Math. 
  \item \lstinline+\usepackage{amsmath}+
  \item Let's now move on to matrices. 
  \end{itemize}
\end{frame}

\begin{frame}[fragile]
  \frametitle{Matrices}
  \begin{itemize}
  \item \lstinline+\bmatrix+ is used to typeset the matrix A
  \item It works similar to the tabular environment
  \item \lstinline+&+ for demarcating columns
  \item \lstinline+\\+ for demarcating rows
  \item Other matrix environments
  \begin{table}
    \center
    \begin{tabular}{c|c}
      \lstinline+matrix+  &  none\\
      \lstinline+pmatrix+ &  \lstinline+(+\\
      \lstinline+Bmatrix+ &  \lstinline+{+\\
        \lstinline+vmatrix+ &  \lstinline+|+\\  
        \lstinline+Vmatrix+ &  \lstinline+||+
    \end{tabular}
  \end{table}
  \end{itemize}
  \tiny See rev19 of \typ{hg}    
\end{frame}

\begin{frame}[fragile]
  \frametitle{Superscripts \& Subscripts}
  \begin{itemize}
  \item \lstinline+^+ for superscripts
  \item \lstinline+_+ for subscripts
  \item Enclose multiple characters in \lstinline+{ }+
  \end{itemize}
\end{frame}

\begin{frame}[fragile]
  \frametitle{Summation \& integration}
  \begin{itemize}
  \item \lstinline+\sum+ command gives the summation symbol
  \item The upper and lower limits are specified using the
    \lstinline+^+ and \lstinline+_+ symbols. 
  \item Similarly the integral symbol is obtained using
    \lstinline+\int+ command. 
  \end{itemize}
\end{frame}

\begin{frame}[fragile]
  \frametitle{\lstinline+displayed+ math}
  \begin{itemize}
  \item Display equations are the other type of displaying math
  \item \LaTeX~ or \lstinline+amsmath+ has a number of environments
    for ``displaying'' equations, with minor differences. 
  \item In general, enclose math in \lstinline+\[+ and \lstinline+\]+
    to get displayed math. 
  \item \lstinline+\begin{equation*}+ is equivalent to this.
  \item Use \lstinline+\begin{equation}+ to get numbered
    equations. %%\end{equation} 
  \end{itemize}
  \tiny See rev20 of \typ{hg}    
\end{frame}

\begin{frame}[fragile]
  \frametitle{Groups of equations}
  \begin{itemize}
  \item The \lstinline+equation+ environment allows typesetting of
    just 1 equation. 
  \item \lstinline+eqnarray+ allows typesetting of multiple equations 
  \item It is similar to the \lstinline+table+ environment
  \item The parts of the equation that need to be aligned are
    indicated using \& symbol.
  \item Each equation is separated by a \lstinline+\newline+ command
  \end{itemize}
  \tiny See rev21, 22 of \typ{hg}    
\end{frame}

\begin{frame}[fragile]
  \frametitle{Fractions \& Surds}
  \begin{itemize}
  \item Fractions are typeset using \lstinline+\frac+ command 
  \item \lstinline+\frac{numerator}{denominator}+ is typeset as
    $\frac{numerator}{denominator}$
  \item Surds are typeset using \lstinline+\sqrt[n]+ command
  \end{itemize}
\end{frame}

\begin{frame}[fragile]
  \frametitle{Greek characters \& Spacing}
  \begin{itemize}
  \item Typesetting Greek characters is simple
  \item \lstinline+\alpha+, \lstinline+\beta+, \lstinline+\gamma+,
    \ldots \lstinline+\Alpha+, \lstinline+\Beta+, \lstinline+\Gamma+
    \ldots 
  \item To get additional spacing in Math environments ---
\begin{center}
\begin{tabular}{|l|l|l|}
\hline
 Abbrev. & Spelled out & Example  \\
\hline
 \lstinline+\,+ & \lstinline+\thinspace+ & $A\,B$ \\
\hline
 \lstinline+\:+ & \lstinline+\medspace+ & $A\:B$ \\
\hline
 \lstinline+\;+ & \lstinline+\thickspace+ & $A\;B$ \\
\hline
   & \lstinline+\quad+ & $A \quad B$ \\
\hline
   & \lstinline+\qquad+ & $A \qquad B$ \\
\hline
 \lstinline+\!+ & \lstinline+\negthinspace+ & $A!B$ \\
\hline
   & \lstinline+\negmedspace+ & $A \negmedspace B$ \\
\hline
   & \lstinline+\negthickspace+ & $A \negthickspace B$ \\
\hline

\end{tabular}
\end{center}
  \end{itemize}
\end{frame}

\section{Bibliography}
\begin{frame}[fragile]
  \frametitle{Bibliography}
  \begin{itemize}
  \item \lstinline+thebibliography+ environment provides a clean and
    simple way to add a bibliography to \LaTeX documents. 
  \item \lstinline+\begin{thebibliography}+ takes as argument the
    maximum width of the label that references will have. 
  \item Each item of the Bibliography is similar to an item in a
    list. 
  \item \lstinline+\bibitem[label]{name}+ followed by the actual
    reference info. 
  \item label replaces auto enumeration numbers 
  \item \lstinline+\cite{name}+ is used to \lstinline+cite+ the
    \lstinline+bibitem+ 
  \item You will need to compile twice. 
  \end{itemize}
  \tiny See rev23 of \typ{hg}    
\end{frame}

\section{Presentations - Beamer}
\begin{frame}[fragile]
  \frametitle{Beamer}
  \begin{itemize}
  \item Use beamer since your report's \LaTeX~ would be re-usable.
  \item It is recommended to start with one of the beamer templates.
  \item Let's look at speaker introduction template.
  \item \lstinline+\documentclass{beamer}+ tells \LaTeX~ to start a
    beamer presentation. 
  \item A beamer document is very similar to any other \LaTeX~
    document except that content is divided into slides. 
  \end{itemize}
\end{frame}

\begin{frame}[fragile]
  \frametitle{Beamer \ldots}
  \begin{itemize}
  \item \lstinline+\usetheme+ command is used to specify the theme of the
    presentation. 
  \item \lstinline+\usecolortheme+ command is used to specify the color
    theme. 
  \item The content of a slide is enclosed within
    \lstinline+\begin{frame}{Title}{Subtitle}+ and
    \lstinline+\end{frame}+ 
  \item If the slide contains \lstinline+verbatim+
    \lstinline+lstlisting+ environments, the \lstinline+\begin{frame}+
    should be passed an additional argument \lstinline+[fragile]+
  \item Overlays can be achieved using the \lstinline+\pause+
    command. 
  \item To achieve more with beamer, it is highly recommended that you
    look at the \texttt{beameruserguide} 
  \end{itemize}
\end{frame}

\begin{frame}[fragile]
  \frametitle{}
  \begin{center}
    \Huge{Thank You!}
  \end{center}
\end{frame}


\end{document} 
 
