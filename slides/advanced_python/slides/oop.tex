\section{Object Oriented Programming}

\begin{frame}[fragile]
  \frametitle{Objectives}
  At the end of this section, you will be able to -
  \begin{itemize}
  \item Understand the differences between Object Oriented Programming
    and Procedural Programming
  \item Appreciate the need for Object Oriented Programming
  \item Read and understand Object Oriented Programs
  \item Write simple Object Oriented Programs
  \end{itemize}
\end{frame}

\begin{frame}[fragile]
  \frametitle{Example: Managing Talks}
  \begin{itemize}
  \item A list of talks at a conference
  \item We want to manage the details of the talks
  \end{itemize}
  \begin{lstlisting}
    talk = {'Speaker': 'Guido van Rossum', 
            'Title': 'The History of Python'
            'Tags': 'python,history,C,advanced'} 

    def get_first_name(talk):
        return talk['Speaker'].split()[0]

    def get_tags(talk):
        return talk['Tags'].split(',')
  \end{lstlisting}
  \begin{itemize}
  \item Not convenient to handle large number of talks
  \end{itemize}
\end{frame}

\begin{frame}[fragile]
  \frametitle{Objects and Methods}
  \begin{itemize}
  \item Objects group data with the procedures/functions
  \item A single entity called \texttt{object}
  \item Everything in Python is an object
  \item Strings, Lists, Functions and even Modules
  \end{itemize}
  \begin{lstlisting}
    s = "Hello World"
    s.lower()

    l = [1, 2, 3, 4, 5]
    l.append(6)
  \end{lstlisting}
\end{frame}

\begin{frame}[fragile]
  \frametitle{Objects \ldots}
  \begin{itemize}
  \item Objects provide a consistent interface
  \end{itemize}
  \begin{lstlisting}
    for element in (1, 2, 3):
        print element
    for key in {'one':1, 'two':2}:
        print key
    for char in "123":
        print char
    for line in open("myfile.txt"):
        print line
    for line in urllib2.urlopen('http://site.com'):
        print line
  \end{lstlisting}
\end{frame}

\begin{frame}[fragile]
  \frametitle{Classes}
  \begin{itemize}
  \item A new string, comes along with methods
  \item A template or a blue-print, where these definitions lie
  \item This blue print for building objects is called a
    \texttt{class}
  \item \texttt{s} is an object of the \texttt{str} class
  \item An object is an ``instance'' of a class
  \end{itemize}
  \begin{lstlisting}
    s = "Hello World"
    type(s)
  \end{lstlisting}
\end{frame}

\begin{frame}[fragile]
  \frametitle{Defining Classes}
  \begin{itemize}
  \item A class equivalent of the talk dictionary
  \item Combines data and methods into a single entity
  \end{itemize}
  \begin{lstlisting}
    class Talk:
        """A class for the Talks."""

        def __init__(self, speaker, title, tags):
            self.speaker = speaker
            self.title = title
            self.tags = tags

        def get_speaker_firstname(self):
            return self.speaker.split()[0]

        def get_tags(self):
            return self.tags.split(',')
  \end{lstlisting}
\end{frame}

\begin{frame}[fragile]
  \frametitle{\texttt{class} block}
  \begin{itemize}
  \item Defined just like a function block
  \item \texttt{class} is a keyword
  \item \texttt{Talk} is the name of the class
  \item Classes also come with doc-strings
  \item All the statements of within the class are inside the block
  \end{itemize}
  \begin{lstlisting}
    class Talk:
        """A class for the Talks."""

        def __init__(self, speaker, title, tags):
            self.speaker = speaker
            self.title = title
            self.tags = tags
  \end{lstlisting}
\end{frame}

\begin{frame}[fragile]
  \frametitle{\texttt{self}}
  \begin{itemize}
  \item Every method has an additional first argument, \texttt{self}
  \item \texttt{self} is a reference to the object itself, of which
    the method is a part of
  \item Variables of the class are referred to as \texttt{self.variablename}
  \end{itemize}
  \begin{lstlisting}
        def get_speaker_firstname(self):
            return self.speaker.split()[0]

        def get_tags(self):
            return self.tags.split(',')
  \end{lstlisting}
\end{frame}

\begin{frame}[fragile]
  \frametitle{Instantiating a Class}
  \begin{itemize}
  \item Creating objects or instances of a class is simple
  \item We call the class name, with arguments as required by it's
    \texttt{\_\_init\_\_} function. 
  \end{itemize}
  \begin{lstlisting}
    bdfl = Talk('Guido van Rossum', 
                'The History of Python', 
                'python,history,C,advanced')
  \end{lstlisting}
  \begin{itemize}
  \item We can now call the methods of the Class
  \end{itemize}
  \begin{lstlisting}
    bdfl.get_tags()
    bdfl.get_speaker_firstname()
  \end{lstlisting}
\end{frame}

\begin{frame}[fragile]
  \frametitle{\texttt{\_\_init\_\_} method}
  \begin{itemize}
  \item A special method 
  \item Called every time an instance of the class is created
  \end{itemize}
  \begin{lstlisting}
    print bdfl.speaker
    print bdfl.tags
    print bdfl.title
  \end{lstlisting}
\end{frame}

\begin{frame}[fragile, allowframebreaks]
  \frametitle{Inheritance}
  \begin{itemize}
  \item Suppose, we wish to write a \texttt{Tutorial} class
  \item It's almost same as \texttt{Talk} except for minor differences
  \item We can ``inherit'' from \texttt{Talk}
  \end{itemize}
  \begin{lstlisting}
    class Tutorial(Talk):
        """A class for the tutorials."""

        def __init__(self, speaker, title, tags, handson=True):
            Talk.__init__(self, speaker, title, tags)
            self.handson = handson

        def is_handson(self):
            return self.handson
  \end{lstlisting}
  \begin{itemize}
  \item Modified \texttt{\_\_init\_\_} method
  \item New \texttt{is\_handson} method
  \item It also has, \texttt{get\_tags} and
    \texttt{get\_speaker\_firstname} 
  \end{itemize}
  \begin{lstlisting}
    numpy = Tutorial('Travis Oliphant', 
                     'Numpy Basics', 
                     'numpy,python,beginner')
    numpy.is_handson()
    numpy.get_speaker_firstname()
  \end{lstlisting}
\end{frame}

\begin{frame}[fragile]
  \frametitle{Summary}
  In this section we have learnt,
  \begin{itemize}
  \item the fundamental difference in paradigm, between Object Oriented
    Programming and Procedural Programming
  \item to write our own classes
  \item to write new classes that inherit from existing classes
  \end{itemize}
\end{frame}

