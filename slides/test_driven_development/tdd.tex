\documentclass[12pt,presentation]{beamer}
\usepackage[utf8]{inputenc}
\usepackage[T1]{fontenc}
\usepackage{fixltx2e}
\usepackage{graphicx}
\usepackage{longtable}
\usepackage{float}
\usepackage{wrapfig}
\usepackage{soul}
\usepackage{textcomp}
\usepackage{marvosym}
\usepackage{wasysym}
\usepackage{latexsym}
\usepackage{amssymb}
\usepackage{hyperref}
\tolerance=1000
\usepackage[english]{babel} \usepackage{ae,aecompl}
\usepackage{mathpazo,courier,euler} \usepackage[scaled=.95]{helvet}
\usepackage{listings}
\lstset{language=Python, basicstyle=\ttfamily\bfseries,
commentstyle=\color{red}\itshape, stringstyle=\color{green},
showstringspaces=false, keywordstyle=\color{blue}\bfseries}
\providecommand{\alert}[1]{\textbf{#1}}

\title{SEES: Test Driven Development}
\author{FOSSEE}

\usetheme{Warsaw}\usecolortheme{default}\useoutertheme{infolines}\setbeamercovered{transparent}

\AtBeginSection[]
{
  \begin{frame}<beamer>
    \frametitle{Outline}
    \tableofcontents[currentsection]
  \end{frame}
}

\begin{document}

\maketitle

\begin{frame}
\frametitle{Outline}
\setcounter{tocdepth}{3}
\tableofcontents
\end{frame}

\section{Introduction}

\begin{frame}
  \frametitle{Objectives}
  At the end of this section, you will be able to:
  \begin{itemize}
  \item Write your code using the TDD paradigm.
  \item Use doctests to test your Python code.
  \item Use unittests to test your Python code.
  \item Use the nose module to test your code.
  \end{itemize}
\end{frame}

\begin{frame}
  \frametitle{What is TDD?}
  The basic steps of TDD are roughly as follows --
  \begin{enumerate}
  \item Decide upon the feature to implement and the methodology of
    testing it.
  \item Write the tests for the feature decided upon.
  \item Just write enough code, so that the test can be run, but it fails.
  \item Improve the code, to just pass the test and at the same time
    passing all previous tests.
  \item Run the tests to see, that all of them run successfully.
  \item Refactor the code you've just written -- optimize the algorithm,
    remove duplication, add documentation, etc.
  \item Run the tests again, to see that all the tests still pass.
  \item Go back to 1.
  \end{enumerate}
\end{frame}

\section{First Test}

\begin{frame}[fragile]
  \frametitle{First Test -- GCD}
  \begin{itemize}
  \item simple program -- GCD of two numbers
  \item What are our code units?
    \begin{itemize}
    \item Only one function \texttt{gcd}
    \item Takes two numbers as arguments
    \item Returns one number, which is their GCD
    \end{itemize}
\begin{lstlisting}
c = gcd(44, 23)
\end{lstlisting}
  \item c will contain the GCD of the two numbers.
  \end{itemize}
\end{frame}

\begin{frame}[fragile]
  \frametitle{Test Cases}
  \begin{itemize}
  \item Important to have test cases and expected outputs even before
    writing the first test!
  \item $a=48$, $b=48$, $GCD=48$
  \item $a=44$, $b=19$, $GCD=1$
  \item Tests are just a series of assertions
  \item True or False, depending on expected and actual behavior
  \end{itemize}

\end{frame}

\begin{frame}[fragile]
  \frametitle{Test Cases -- Code}
\begin{lstlisting}
tc1 = gcd(48, 64)
if tc1 != 16:
    print "Failed for a=48, b=64. Expected 16. \
    Obtained %d instead." % tc1
    exit(1)

tc2 = gcd(44, 19)
if tc2 != 1:
    print "Failed for a=44, b=19. Expected 1. \
    Obtained %d instead." % tc2
    exit(1)

print "All tests passed!"
\end{lstlisting}
\begin{itemize}
\item The function \texttt{gcd} doesn't even exist!
\end{itemize}
\end{frame}


\begin{frame}[fragile]
  \frametitle{Stubs}
  \begin{itemize}
  \item First write a very minimal definition of \texttt{gcd}
    \begin{lstlisting}
def gcd(a, b):
    pass
    \end{lstlisting}
  \item Written just, so that the tests can run
  \item Obviously, the tests are going to fail
  \end{itemize}
\end{frame}

\begin{frame}[fragile]
  \frametitle{\texttt{gcd.py}}
\begin{lstlisting}
def gcd(a, b):
    pass

if __name__ == '__main__':
    tc1 = gcd(48, 64)
    if tc1 != 16:
        print "Failed for a=48 and b=64. \
        Expected 16. Obtained %d instead." % tc1
        exit(1)
    tc2 = gcd(44, 19)
    if tc2 != 1:
        print "Failed for a=44 and b=19. \
        Expected 1. Obtained %d instead." % tc2
        exit(1)
    print "All tests passed!"
\end{lstlisting}
\end{frame}

\begin{frame}[fragile]
  \frametitle{First run}
\begin{lstlisting}
$ python gcd.py
Traceback (most recent call last):
  File "gcd.py", line 7, in <module> 
  print "Failed for a=48 and b=64. Expected 16. 
  Obtained %d instead." % tc1
TypeError: %d format: 
a number is required, not NoneType
\end{lstlisting} %$

  \begin{itemize}
  \item We have our code unit stub, and a failing test. 
  \item The next step is to write code, so that the test just passes.
  \end{itemize}
\end{frame}

\begin{frame}[fragile]
  \frametitle{Euclidean Algorithm}
  \begin{itemize}
  \item Modify the \texttt{gcd} stub function
  \item Then, run the script to see if the tests pass.
  \end{itemize}
\begin{lstlisting}
def gcd(a, b):
    if a == 0:
        return b
    while b != 0:
        if a > b:
            a = a - b
        else:
            b = b - a
    return a
\end{lstlisting}
\begin{lstlisting}
$ python gcd.py
All tests passed!
\end{lstlisting} %$
  \begin{itemize}
  \item \alert{Success!}
  \end{itemize}
\end{frame}


\begin{frame}[fragile]
  \frametitle{Euclidean Algorithm -- Modulo}
  \begin{itemize}
  \item Repeated subtraction can be replaced by a modulo
  \item modulo of \texttt{a\%b} is always less than b
  \item when \texttt{a < b}, \texttt{a\%b} equals \texttt{a}
  \item Combine these two observations, and modify the code
\begin{lstlisting}
def gcd(a, b):
    while b != 0:
        a, b = b, a % b
    return a
\end{lstlisting}
  \item Check that the tests pass again
  \end{itemize}
\end{frame}

\begin{frame}[fragile]
  \frametitle{Euclidean Algorithm -- Recursive}
  \begin{itemize}
  \item Final improvement -- make \texttt{gcd} recursive
  \item More readable and easier to understand
\begin{lstlisting}
def gcd(a, b):
    if b == 0:
        return a
    return gcd(b, a%b)
\end{lstlisting}
  \item Check that the tests pass again
  \end{itemize}
\end{frame}

\begin{frame}[fragile]
  \frametitle{Document \texttt{gcd}}
  \begin{itemize}
  \item Undocumented function is as good as unusable
  \item Let's add a docstring \& We have our first test!
  \end{itemize}
\begin{lstlisting}
def gcd(a, b):
    """Returns the Greatest Common Divisor of the 
    two integers passed as arguments.

    Args:
      a: an integer
      b: another integer

    Returns: Greatest Common Divisor of a and b
    """
    if b == 0:
        return a
    return gcd(b, a%b)
\end{lstlisting}
\end{frame}


\begin{frame}[fragile]
  \frametitle{Persistent Test Cases}
  \begin{itemize}
  \item Tests should be pre-determined and written, before the code
  \item Test Data is repeatedly used; Hence, saved in persistent
    format
  \item Let's save data for GCD tests in a text file. 
  \item The file shall have multiple lines of test data
  \item Each line corresponds to a single test case
  \item Each line consists of three comma separated values --
    \begin{itemize}
    \item First two coloumns are the integers for which the GCD has to be
      computed
    \item Third coloumn is the expected GCD to the preceding two
      numbers.
    \end{itemize}
  \item Let us call our data file \texttt{gcd\_testcases.dat}
  \end{itemize}
\end{frame}

\begin{frame}[fragile]
  \frametitle{Modify \texttt{gcd.py}}
\begin{lstlisting}
if __name__ == '__main__':
    for line in open('gcd_testcases.dat'):
        values = line.split(', ')
        a = int(values[0])
        b = int(values[1])
        g = int(values[2])

        tc = gcd(a, b)
        if tc != g:
            print "Failed for a=%d and b=%d.\
            Expected %d. Obtained %d instead."\
            % (a, b, g, tc)
            exit(1)

    print "All tests passed!"
\end{lstlisting}
\end{frame}

\section{Python Testing Frameworks}

\begin{frame}[fragile]
  \frametitle{Python Testing Frameworks}
  \begin{itemize}
  \item Testing frameworks essentially, ease the job of the user
  \item Python provides two frameworks for testing code
    \begin{itemize}
    \item \texttt{unittest} framework
    \item \texttt{doctest} module
    \end{itemize}
  \end{itemize}
\end{frame}

\subsection{\texttt{doctest} module}

\begin{frame}[fragile]
  \frametitle{doctest}
  \begin{itemize}
  \item Documentation always accompanies a well written piece of code
  \item Use \texttt{docstring} to document functions or modules
  \item Along with description and usage, examples can be added
  \item Interactive interpreter session inputs and outputs are
    copy-pasted 
  \item \texttt{doctest} module picks up all such interactive examples
  \item Executes them and determines if the code runs, as documented
  \end{itemize}
  Let's use the \texttt{doctest} module for our \texttt{gcd} function
\end{frame}


\begin{frame}[fragile]
  \frametitle{doctest for \texttt{gcd.py}}
\begin{tiny}
\begin{lstlisting}
def gcd(a, b):
    """Returns the Greatest Common Divisor of the two integers
    passed as arguments.

    Args:
      a: an integer
      b: another integer

    Returns: Greatest Common Divisor of a and b

    >>> gcd(48, 64)
    16
    >>> gcd(44, 19)
    1
    """
    if b == 0:
        return a
    return gcd(b, a%b) 
\end{lstlisting}
\end{tiny}
\begin{itemize}
\item We have added examples to the \texttt{docstring}
\item Now we need to tell the \texttt{doctest} module to execute
\end{itemize}
\end{frame}

\begin{frame}[fragile]
  \frametitle{doctest for \texttt{gcd.py} \ldots}
\begin{lstlisting}
if __name__ == "__main__":
    import doctest
    doctest.testmod()
\end{lstlisting}
\begin{itemize}
\item \texttt{testmod} automatically picks all sample sessions
\item Executes them and compares output with documented output
\item It doesn't give any output, when all the results match 
\item Complains only when one or more tests fail. 
\end{itemize}
\end{frame}

\begin{frame}[fragile]
  \frametitle{\texttt{doctest} -- Execution}
  \begin{itemize}
  \item Run the doctests by running \texttt{gcd.py}
\begin{lstlisting}
$ python gcd.py
\end{lstlisting} %$
  \item All the tests pass, and doctest gives no output
  \item For a more detailed report we can run with -v argument
\begin{lstlisting}
$ python gcd.py -v
\end{lstlisting} %$
  \item If the output contains a blank line, use \texttt{<BLANKLINE>}
  \item To see a failing test case, replace \texttt{return a} with \texttt{b}
  \end{itemize}
\end{frame}

\subsection{\texttt{unittest} framework}

\begin{frame}[fragile]
  \frametitle{\texttt{unittest}}
  \begin{itemize}
  \item It won't be long, before we complain about the power of
    \texttt{doctest} 
  \item \texttt{unittest} framework can efficiently automate tests
  \item Easily initialize code and data for executing the specific
    tests
  \item Cleanly shut them down once the tests are executed
  \item Easily aggregate tests into collections and improved reporting
  \end{itemize}
\end{frame}


\begin{frame}[fragile]
  \frametitle{\texttt{unittest}}
  \begin{itemize}
  \item It won't be long, before we complain about the power of
    \texttt{doctest} 
  \item \texttt{unittest} framework can efficiently automate tests
  \item Easily initialize code and data for executing the specific
    tests
  \item Cleanly shut them down once the tests are executed
  \item Easily aggregate tests into collections and improved reporting
  \end{itemize}
\end{frame}

\begin{frame}[fragile]
  \frametitle{\texttt{unittest}ing \texttt{gcd.py}}
  \begin{itemize}
  \item Subclass the \texttt{TestCase} class in \texttt{unittest}
  \item Place all the test code as methods of this class
  \item Use the test cases present in \texttt{gcd\_testcases.dat}
  \item Place the code in \texttt{test\_gcd.py} 
  \end{itemize}
\end{frame}

\begin{frame}[fragile,allowframebreaks]
  \frametitle{\texttt{test\_gcd.py}}
\small
\begin{lstlisting}
import gcd
import unittest

class TestGcdFunction(unittest.TestCase):

    def setUp(self):
        self.test_file = open('gcd_testcases.dat')
        self.test_cases = []
        for line in self.test_file:
            values = line.split(', ')
            a = int(values[0])
            b = int(values[1])
            g = int(values[2])

            self.test_cases.append([a, b, g])

    def test_gcd(self):
        for case in self.test_cases:
            a = case[0]
            b = case[1]
            g = case[2]
            self.assertEqual(gcd.gcd(a, b), g)

    def tearDown(self):
        self.test_file.close()
        del self.test_cases

if __name__ == '__main__':
    unittest.main()
\end{lstlisting}
\end{frame}

\begin{frame}[fragile]
  \frametitle{\texttt{test\_gcd.py}}
  \begin{itemize}
  \item \texttt{setUp} -- we read all the test data and store it in a
    list
  \item \texttt{tearDown} -- delete the data to free up memory and
    close open file
  \item \texttt{test\_gcd} -- actual test code
  \item \texttt{assertEqual} -- compare actual result with expected one
  \item Write documentation for above code. 
  \end{itemize}
\end{frame}

\section{\texttt{nose}}

\begin{frame}[fragile]
  \frametitle{\texttt{nose} tests}
  \begin{itemize}
  \item It is not easy to organize, choose and run tests scattered
    across multiple files. 
  \item \texttt{nose} module aggregate these tests automatically
  \item Can aggregate \texttt{unittests} and \texttt{doctests}
  \item Allows us to pick and choose which tests to run
  \item Helps output the test-results and aggregate them in various
    formats
  \item Not part of the Python distribution itself
\begin{lstlisting}
$ easy_install nose
\end{lstlisting} %$
  \item Run the following command in the top level directory
\begin{lstlisting}
$ nosetests
\end{lstlisting} %$
  \end{itemize}
\end{frame}

\end{document}
