\documentclass[12pt,presentation]{beamer}
\usepackage[utf8]{inputenc}
\usepackage[T1]{fontenc}
\usepackage{fixltx2e}
\usepackage{graphicx}
\usepackage{longtable}
\usepackage{float}
\usepackage{wrapfig}
\usepackage{soul}
\usepackage{textcomp}
\usepackage{marvosym}
\usepackage{wasysym}
\usepackage{latexsym}
\usepackage{amssymb}
\usepackage{hyperref}
\tolerance=1000
\usepackage[english]{babel} \usepackage{ae,aecompl}
\usepackage{mathpazo,courier,euler} \usepackage[scaled=.95]{helvet}
\usepackage{listings}
\lstset{language=Python, basicstyle=\ttfamily\bfseries,
commentstyle=\color{red}\itshape, stringstyle=\color{green},
showstringspaces=false, keywordstyle=\color{blue}\bfseries}
\providecommand{\alert}[1]{\textbf{#1}}

\title{More Basic Python}
\author[FOSSEE] {FOSSEE}
\institute[IIT Bombay] {Department of Aerospace Engineering\\IIT Bombay}
\date []{}

\usetheme{Warsaw}\usecolortheme{default}\useoutertheme{infolines}\setbeamercovered{transparent}

\AtBeginSection[]
{
  \begin{frame}<beamer>
    \frametitle{Outline}
    \tableofcontents[currentsection]
  \end{frame}
}

\begin{document}

\maketitle

\begin{frame}
\frametitle{Outline}
\setcounter{tocdepth}{3}
\tableofcontents
\end{frame}

\section{Using Python modules}

\begin{frame}[fragile]
  \frametitle{\texttt{hello.py}}
    \begin{itemize}
    \item Script to print `hello world' -- \texttt{hello.py}
    \end{itemize}
    \begin{lstlisting}
     print "Hello world!"
    \end{lstlisting}
    \begin{itemize}
    \item We have been running scripts from IPython
    \end{itemize}
    \begin{lstlisting}
     In[]: %run -i hello.py
    \end{lstlisting}
    \begin{itemize}
    \item Now, we run from the shell using python
    \end{itemize}
    \begin{lstlisting}
      $ python hello.py
    \end{lstlisting} 
\end{frame}


\begin{frame}
  \frametitle{Modules}
  \begin{itemize}
      \item Organize your code
      \item Collect similar functionality
      \item Functions, classes, constants, etc.
  \end{itemize}
\end{frame}

\begin{frame}
  \frametitle{Modules}
  \begin{itemize}
  \item Define variables, functions and classes in a file with a
    \texttt{.py} extension
  \item This file becomes a module!
  \item The \texttt{import} keyword ``loads'' a module
  \item One can also use:
    \mbox{\texttt{from module import name1, name2, name2}}\\
    where \texttt{name1} etc. are names in the module, ``module''
  \item \texttt{from module import *} \ --- imports everything from module,
    \alert{use only in interactive mode}
    \item File name should be valid variable name
  \end{itemize}
\end{frame}

\begin{frame}[fragile]
  \frametitle{Modules: example}
  \begin{lstlisting}
# --- foo.py ---
some_var = 1
def fib(n): # write Fibonacci series up to n
    """Print a Fibonacci series up to n."""
    a, b = 0, 1
    while b < n:
        print b,
        a, b = b, a+b
# EOF
  \end{lstlisting}
\end{frame}

\begin{frame}[fragile]
  \frametitle{Modules: example}
  \begin{lstlisting}
>>> import foo
>>> foo.fib(10)
1 1 2 3 5 8 
>>> foo.some_var
1
  \end{lstlisting}
\end{frame}


\begin{frame}[fragile]
  \frametitle{Python path}
  \begin{itemize}
  \item In IPython type the following
  \end{itemize}
  \begin{lstlisting}
    import sys
    sys.path
  \end{lstlisting}
  \begin{itemize}
  \item List of locations where python searches for a module
  \item \texttt{import sys} -- searches for file \texttt{sys.py} or
    dir \texttt{sys} in all these locations
  \item So, our own modules can be in any one of the locations
  \item Current working directory is one of the locations
  \item Can also set \texttt{PYTHONPATH} env var
  \end{itemize}
\end{frame}


\begin{frame}[fragile]
  \frametitle{Another example: GCD script}
  \begin{itemize}
  \item Function that computes gcd of two numbers
  \item Save it as \texttt{gcd\_script.py}
  \end{itemize}
  \begin{lstlisting}
    def gcd(a, b):
        while b:
            a, b = b, a%b
        return a
  \end{lstlisting}
  \begin{itemize}
  \item Also add the tests to the file
  \end{itemize}
  \begin{lstlisting}
    if gcd(40, 12) == 4 and gcd(12, 13) == 1:
        print "Everything OK"
    else:
        print "The GCD function is wrong"
  \end{lstlisting}
  \begin{lstlisting}
    $ python gcd_script.py
  \end{lstlisting} % $
\end{frame}

\begin{frame}[fragile]
  \frametitle{\texttt{\_\_name\_\_}}
  \begin{lstlisting}
    import gcd_script
  \end{lstlisting}
  \begin{itemize}
  \item The import is successful
  \item But the test code, gets run
  \item Add the tests to the following \texttt{if} block
  \end{itemize}
  \begin{lstlisting}
    if __name__ == "__main__":
  \end{lstlisting}
  \begin{itemize}
  \item Now the script runs properly 
  \item As well as the import works; test code not executed
  \item \texttt{\_\_name\_\_} is local to every module and is equal
    to \texttt{\_\_main\_\_} only when the file is run as a script.
  \end{itemize}
\end{frame}


\begin{frame}[fragile]
  \frametitle{Stand-alone scripts}
Consider a file \texttt{f.py}:
\footnotesize
\begin{lstlisting}
#!/usr/bin/env python
"""Module level documentation."""
# First line tells the shell that it should use Python
# to interpret the code in the file.
def f():
    print "f"

# Check if we are running standalone or as module.
# When imported, __name__ will not be '__main__'
if __name__ == '__main__':
    # This is not executed when f.py is imported.
    f() 
\end{lstlisting}
\end{frame}


\section{Exceptions}

\begin{frame}{Motivation}
    \begin{itemize}
        \item How do you signal errors to a user?
    \end{itemize}
\end{frame}

\begin{frame}
  \frametitle{Exceptions}
  \begin{itemize}
  \item Python's way of notifying you of errors
  \item Several standard exceptions: \texttt{SyntaxError}, \texttt{IOError}
    etc.
  \item Users can also \texttt{raise} errors
  \item Users can create their own exceptions
  \item Exceptions can be ``caught'' via \texttt{try/except} blocks
  \end{itemize}
\end{frame}

\begin{frame}[fragile]
  \frametitle{Exception: examples}
\begin{lstlisting}
>>> 10 * (1/0)
Traceback (most recent call last):
  File "<stdin>", line 1, in ?
ZeroDivisionError: integer division or modulo by zero
>>> 4 + spam*3
Traceback (most recent call last):
  File "<stdin>", line 1, in ?
NameError: name 'spam' is not defined
>>> '2' + 2
Traceback (most recent call last):
  File "<stdin>", line 1, in ?
TypeError: cannot concatenate 'str' and 'int' objects
\end{lstlisting}
\end{frame}

\begin{frame}[fragile]
  \frametitle{Exception: examples}
\begin{lstlisting}
>>> while True:
...     try:
...         x = int(raw_input("Enter a number: "))
...         break
...     except ValueError:
...         print "Invalid number, try again..."
...
>>> # To raise exceptions
... raise ValueError, "your error message"
Traceback (most recent call last):
  File "<stdin>", line 2, in ?
ValueError: your error message
\end{lstlisting}
\end{frame}

\begin{frame}[fragile]
  \frametitle{Exception: try/finally}
\begin{lstlisting}
>>> while True:
...     try:
...         x = open("my_data.txt")
...         lines = x.readlines()
...         # Process the data from the file.
...         value = int(line[0])
...     except ValueError:
...         print "Invalid file!"
...     finally:
...         print "All good!"
...
\end{lstlisting}
\end{frame}



\end{document}
