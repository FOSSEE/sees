\section{Strings}

\begin{frame}[fragile]
  \frametitle{What are Strings?}
  \begin{itemize}
  \item Anything quoted is a string
  \item Single quotes, double quotes or triple single/double quotes
  \item Any length --- single character, null string, \ldots
  \end{itemize}
  \begin{lstlisting}
   In[]: 'This is a string'
   In[]: "This is a string too'
   In[]: '''This is a string as well'''
   In[]: """This is also a string"""
   In[]: ''  # empty string
  \end{lstlisting}
\end{frame}

\begin{frame}[fragile]
  \frametitle{Why so many?}
  \begin{itemize}
  \item Reduce the need for escaping 
  \end{itemize}
  \begin{lstlisting}
   In[]: "Python's strings are powerful!"
   In[]: 'He said, "I love Python!"'
  \end{lstlisting}
  \begin{itemize}
  \item Triple quoted strings can be multi-line
  \item Used for doc-strings
  \end{itemize}
\end{frame}

\begin{frame}[fragile]
  \frametitle{Assignment \& Operations}
  \begin{lstlisting}
   In[]: a = 'Hello'
   In[]: b = 'World'
   In[]: c = a + ', ' + b + '!'
  \end{lstlisting}
  \begin{itemize}
  \item Strings can be multiplied with numbers
  \end{itemize}
  \begin{lstlisting}
   In[]: a = 'Hello'
   In[]: a * 5
  \end{lstlisting}
\end{frame}

\begin{frame}[fragile]
  \frametitle{Accessing Elements}
  \begin{lstlisting}
   In[]: print a[0], a[4], a[-1], a[-4]
  \end{lstlisting}
  \begin{itemize}
  \item Can we change the elements?
  \end{itemize}
  \begin{lstlisting}
   In[]: a[0] = 'H'
  \end{lstlisting}
  \begin{itemize}
  \item Strings are immutable!
  \end{itemize}
\end{frame}

\begin{frame}[fragile]
  \frametitle{Problem - Day of the Week?}
  \begin{itemize}
  \item Strings have methods to manipulate them
  \end{itemize}
  \begin{block}{Problem}
    Given a list, \texttt{week}, containing names of the days of the
    week and a string \texttt{s}, check if the string is a day of the
    week. We should be able to check for any of the forms like,
    \emph{sat, saturday, Sat, Saturday, SAT, SATURDAY}
  \end{block}
  \begin{itemize}
  \item Get the first 3 characters of the string
  \item Convert it all to lower case
  \item Check for existence in the list, \texttt{week}
  \end{itemize}
\end{frame}

\begin{frame}[fragile]
  \frametitle{Slicing}
  \begin{lstlisting}
   In[]:  q = "Hello World"
   In[]: q[0:3]
   In[]: q[:3]
   In[]: q[3:]
   In[]: q[:]
   In[]: q[-1:1]
   In[]: q[1:-1]
  \end{lstlisting}
  \begin{itemize}
  \item One or both of the limits, is optional
  \end{itemize}
\end{frame}

\begin{frame}[fragile]
  \frametitle{Striding}
  \begin{lstlisting}
   In[]: q[0:5:1]
   In[]: q[0:5:2]
   In[]: q[0:5:3]
   In[]: q[0::2]
   In[]: q[2::2]
   In[]: q[::2]
   In[]: q[5:0:-1]
   In[]: q[::-1]
  \end{lstlisting}
\end{frame}

\begin{frame}[fragile]
  \frametitle{String Methods} 
  \begin{lstlisting}
   In[]: s.lower()
   In[]: s.upper()
    s.<TAB>
  \end{lstlisting}
  \begin{itemize}
  \item \alert{Strings are immutable!}
  \item A new string is being returned
  \end{itemize}
\end{frame}

\begin{frame}[fragile]
  \frametitle{Solution - Day of the Week?}
  \begin{lstlisting}
   In[]: s.lower()[:3] in week
  \end{lstlisting}
  OR
  \begin{lstlisting}
   In[]: s[:3].lower() in week
  \end{lstlisting}
\end{frame}

\begin{frame}[fragile]
  \frametitle{\texttt{join} a list of strings}
  \begin{itemize}
  \item Given a list of strings
  \item We wish to join them into a single string
  \item Possibly, each string separated by a common token
  \end{itemize}
  \begin{lstlisting}
  In[]:  email_list = ["info@fossee.in", 
                  "enquiries@fossee.in",
                  "help@fossee.in"]
  \end{lstlisting}
  \begin{lstlisting}
  In[]:  '; '.join(email_list)
  In[]:  ', '.join(email_list)
  \end{lstlisting}
\end{frame}

\section{Conditionals}

\begin{frame}[fragile]
  \frametitle{\texttt{if-else} block}
  \begin{lstlisting}
  In[]:  a = 5
  In[]:  if a % 2 == 0:
  ....:     print "Even"
  ....:  else:
  ....:     print "Odd"
  \end{lstlisting}
  \begin{itemize}
  \item A code block -- \texttt{:} and indentation
  \item Exactly one block gets executed in the \texttt{if-else}
  \end{itemize}
\end{frame}


\begin{frame}[fragile]
  \frametitle{\texttt{if-elif-else}}
  \begin{lstlisting}
   In[]: if a > 0:           
   ....:     print "positive"
   ....: elif a < 0:         
   ....:     print "negative"
   ....: else:               
   ....:     print "zero"    
  \end{lstlisting}
  \begin{itemize}
  \item Only one block gets executed, depending on \texttt{a}
  \end{itemize}
\end{frame}

\begin{frame}[fragile]
  \frametitle{\texttt{else} is optional}
  \begin{lstlisting}
  In[]:  if user == 'admin':         
  ....:      admin_Operations()
  ....:  elif user == 'moderator':   
  ....:      moderator_operations()
  ....:  elif user == 'client':      
  ....:      customer_operations()
  \end{lstlisting}
  \begin{itemize}
  \item Note that there is no \texttt{else} block
  \end{itemize}
\end{frame}

\begin{frame}[fragile]
  \frametitle{Ternary operator}
  \begin{itemize}
  \item \texttt{score\_str} is either \texttt{'AA'} or a string of one
    of the numbers in the range 0 to 100. 
  \item We wish to convert the string to a number using \texttt{int}
  \item Convert it to 0, when it is \texttt{'AA'}
  \item \texttt{if-else} construct or the ternary operator
  \end{itemize}
  \begin{lstlisting}
   In[]: if score_str != 'AA':
   ....:     score = int(score_str)
   ....: else:
   ....:     score = 0
  \end{lstlisting}
  \begin{lstlisting}
   In[]: ss = score_str
   In[]: score = int(ss) if ss != 'AA' else 0
  \end{lstlisting}
\end{frame}


\begin{frame}[fragile]
  \frametitle{\texttt{pass}}
  \begin{itemize}
  \item \texttt{pass} is a syntactic filler
  \item When a certain block has no statements, a \texttt{pass} is
    thrown in
  \item Mostly, when you want to get back to that part, later.
  \end{itemize}
\end{frame}

\section{Loops}

\begin{frame}[fragile]
  \frametitle{\texttt{while}}
  \begin{itemize}
  \item Print squares of all odd numbers less than 10 using
    \texttt{while} 
  \end{itemize}
  \begin{lstlisting}
  In[]:  i = 1

  In[]:  while i<10:
  ....:      print i*i
  ....:      i += 2
  \end{lstlisting}
  \begin{itemize}
  \item The loops runs as long as the condition is \texttt{True}
  \end{itemize}
\end{frame}

\begin{frame}[fragile]
  \frametitle{\texttt{for}}
  \begin{itemize}
  \item Print squares of all odd numbers less than 10 using
    \texttt{for}
  \end{itemize}
  \begin{lstlisting}
  In[]:  for n in [1, 2, 3]:
  ....:      print n
  \end{lstlisting}
  \begin{itemize}
  \item \texttt{for} iterates over each element of a sequence
  \end{itemize}
  \begin{lstlisting}
  In[]:  for n in [1, 3, 5, 7, 9]: 
  ....:      print n*n

  In[]:  for n in range(1, 10, 2):
  ....:      print n*n
  \end{lstlisting}
  \begin{itemize}
  \item \alert{range([start,] stop[, step])}
  \item Returns a list; Stop value is not included. 
  \end{itemize}
\end{frame}


\begin{frame}[fragile]
  \frametitle{\texttt{break}}
  \begin{itemize}
  \item breaks out of the innermost loop.
  \item Squares of odd numbers below 10 using \texttt{while} \&
    \texttt{break}
  \end{itemize}
  \begin{lstlisting}
  In[]:  i = 1

  In[]:  while True:
  ....:      print i*i
  ....:      i += 2
  ....:      if i>10:
  ....:          break
  \end{lstlisting}
\end{frame}

\begin{frame}[fragile]
  \frametitle{\texttt{continue}}
  \begin{itemize}
  \item Skips execution of rest of the loop on current iteration
  \item Jumps to the end of this iteration
  \item Squares of all odd numbers below 10, not multiples of 3
  \end{itemize}
  \begin{lstlisting}
  In[]:  for n in range(1, 10, 2):
  ....:      if n%3 == 0:
  ....:          continue      
  ....:      print n*n
  \end{lstlisting}
\end{frame}


\section{Lists}

\begin{frame}[fragile]
  \frametitle{Creating Lists}
  \begin{lstlisting}
   In[]: empty = [] 

   In[]: p = ['spam', 'eggs', 100, 1.234]
   In[]: q = [[4, 2, 3, 4], 'and', 1, 2, 3, 4]
  \end{lstlisting}
  \begin{itemize}
  \item Lists can be empty, with no elements in them
  \item Lists can be heterogeneous -- every element of different kind
  \end{itemize}
\end{frame}

\begin{frame}[fragile]
  \frametitle{Accessing Elements}
  \begin{lstlisting}
  In[]:  print p[0], p[1], p[3]

  In[]:  print p[-1], p[-2], p[-4]
  In[]:  print p[10]
  \end{lstlisting}
  \begin{itemize}
  \item Indexing starts from 0
  \item Indexes can be negative
  \item Indexes should be in the valid range
  \end{itemize}
\end{frame}

\begin{frame}[fragile]
  \frametitle{Accessing Elements \& length}
  \begin{lstlisting}
  In[]:  print p[0], p[1], p[3]

  In[]: print p[-1], p[-2], p[-4]
  In[]: print len(p)
  In[]:  print p[10]
  \end{lstlisting}
  \begin{itemize}
  \item Indexing starts from 0
  \item Indexes can be negative
  \item Indexes should be within the \texttt{range(0, len(p))}
  \end{itemize}
\end{frame}


\begin{frame}[fragile]
  \frametitle{Adding \& Removing Elements}
  \begin{itemize}
  \item The append method adds elements to the end of the list
  \end{itemize}
  \begin{lstlisting}
  In[]:  p.append('onemore')
  In[]:  p
  In[]:  p.append([1, 6])
  In[]: p
  \end{lstlisting}
  \begin{itemize}
  \item Elements can be removed based on their index OR
  \item based on the value of the element
  \end{itemize}
  \begin{lstlisting}
   In[]: del p[1]
   In[]: p.remove(100)
  \end{lstlisting}
  \begin{itemize}
  \item \alert{When removing by value, first element is removed}
  \end{itemize}
\end{frame}


\begin{frame}[fragile]
  \frametitle{Concatenating lists}
  \begin{lstlisting}
  In[]:  a = [1, 2, 3, 4]
  In[]:  b = [4, 5, 6, 7]
  In[]:  a + b
  In[]:  print a+b, a, b
  \end{lstlisting}
  \begin{itemize}
  \item A new list is returned; None of the original lists change
  \end{itemize}
  \begin{lstlisting}
   In[]: c = a + b
   In[]: c
  \end{lstlisting}
\end{frame}

\begin{frame}[fragile]
  \frametitle{Slicing \& Striding}
  \begin{lstlisting}
  In[]:  primes = [2, 3, 5, 7, 11, 13, 17, 19, 23, 29]
  In[]:  primes[4:8]
  In[]:  primes[:4]

  In[]:  num = [0, 1, 2, 3, 4, 5, 6, 7, 8, 9, 10, 11, 12, 13]
  In[]:  num[1:10:2]
  In[]:  num[:10]
  In[]: num[10:]
  In[]: num[::2]
  In[]: num[::-1]
  \end{lstlisting}
\end{frame}

\begin{frame}[fragile]
  \frametitle{Sorting}
  \begin{lstlisting}
  In[]:  a = [5, 1, 6, 7, 7, 10]
  In[]:  a.sort()
  In[]:  a
  \end{lstlisting}
  \begin{itemize}
  \item \texttt{sort} method sorts the list in-place
  \item Use \texttt{sorted} if you require a new list
  \end{itemize}
  \begin{lstlisting}
  In[]:  a = [5, 1, 6, 7, 7, 10]
  In[]:  sorted(a)
  In[]:  a
  \end{lstlisting}
\end{frame}

\begin{frame}[fragile]
  \frametitle{Reversing}
  \begin{lstlisting}
  In[]:  a = [5, 1, 6, 7, 7, 10]
  In[]:  a.reverse()
  In[]:  a
  \end{lstlisting}
  \begin{itemize}
  \item \texttt{reverse} method reverses the list in-place
  \item Use \texttt{[::-1]} if you require a new list
  \end{itemize}
  \begin{lstlisting}
  In[]:  a = [5, 1, 6, 7, 7, 10]
  In[]:  a[::-1]
  In[]:  a
  \end{lstlisting}
\end{frame}
