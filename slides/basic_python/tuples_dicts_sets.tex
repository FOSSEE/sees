\section{Tuples}

\begin{frame}[fragile]
  \frametitle{Tuples -- Initialization}
  \begin{lstlisting}
  In[]: t = (1, 2.5, "hello", -4, "world", 1.24, 5)
  In[]: t
  \end{lstlisting}
  \begin{itemize}
  \item It is not always necessary to use parenthesis
  \end{itemize}
  \begin{lstlisting}
  In[]: a = 1, 2, 3 
  In[]: b = 1, 
  \end{lstlisting}
\end{frame}

\begin{frame}[fragile]
  \frametitle{Indexing}
  \begin{lstlisting}
   In[]: t[3]
   In[]: t[1:5:2]
   In[]: t[2] = "Hello"
  \end{lstlisting}
  \begin{itemize}
  \item \alert{Tuples are immutable!}
  \end{itemize}
\end{frame}

\begin{frame}[fragile]
  \frametitle{Swapping values}
  \begin{lstlisting}
  In[]: a = 5
  In[]: b = 7

  In[]: temp = a
  In[]: a = b
  In[]: b = temp
  \end{lstlisting}
  \begin{itemize}
  \item Here's the Pythonic way of doing it
  \end{itemize}
  \begin{lstlisting}
  In[]: a, b = b, a
  \end{lstlisting}
  \begin{itemize}
  \item The variables can be of different data-types
  \end{itemize}
  \begin{lstlisting}
  In[]: a = 2.5
  In[]: b = "hello"
  In[]: a, b = b, a
  \end{lstlisting}
\end{frame}

\begin{frame}[fragile]
  \frametitle{Tuple packing \& unpacking}
  \begin{lstlisting}
   In[]: 5,

   In[]: 5, "hello", 2.5
  \end{lstlisting}
  \begin{itemize}
  \item Tuple packing and unpacking, when swapping
  \end{itemize}
  \begin{lstlisting}
  In[]: a, b = b, a
  \end{lstlisting}
\end{frame}

\section{Dictionaries}

\begin{frame}[fragile]
  \frametitle{Creating Dictionaries}
  \begin{lstlisting}
  In[]: mt_dict = {}

  In[]: extensions = {'jpg' : 'JPEG Image', 
                  'py' : 'Python script', 
                  'html' : 'Html document', 
                  'pdf' : 'Portable Document Format'}

  In[]: extensions
  \end{lstlisting}
  \begin{itemize}
  \item Key-Value pairs
  \item \alert{ No ordering of keys! }
  \end{itemize}
\end{frame}

\begin{frame}[fragile]
  \frametitle{Accessing Elements}
  \begin{lstlisting}
  In[]: print extensions['jpg']
  \end{lstlisting}
  \begin{itemize}
  \item Values can be accessed using keys
  \end{itemize}
  \begin{lstlisting}
  In[]: print extensions['zip']
  \end{lstlisting}
  \begin{itemize}
  \item Values of non-existent keys cannot, obviously, be accessed
  \end{itemize}
\end{frame}

\begin{frame}[fragile]
  \frametitle{Adding \& Removing Elements}
  \begin{itemize}
  \item Adding a new key-value pair
  \end{itemize}
  \begin{lstlisting}
   In[]: extensions['cpp'] = 'C++ code'
   In[]: extensions
  \end{lstlisting}
  \begin{itemize}
  \item Deleting a key-value pair
  \end{itemize}
  \begin{lstlisting}
   In[]: del extension['pdf']
   In[]: extensions
  \end{lstlisting}
  \begin{itemize}
  \item Assigning to existing key, modifies the value
  \end{itemize}
  \begin{lstlisting}
   In[]: extensions['cpp'] = 'C++ source code'
   In[]: extensions
  \end{lstlisting}
\end{frame}

\begin{frame}[fragile]
  \frametitle{Containership}
  \begin{lstlisting}
   In[]: 'py' in extensions
   In[]: 'odt' in extensions
  \end{lstlisting}
  \begin{itemize}
  \item Allow checking for container-ship of keys; NOT values
  \item Use the \texttt{in} keyword to check for container-ship
  \end{itemize}
\end{frame}

\begin{frame}[fragile]
  \frametitle{Lists of Keys and Values}
  \begin{lstlisting}
   In[]: extensions.keys()
   In[]: extensions.values()
  \end{lstlisting}
  \begin{itemize}
  \item Note that the order of the keys and values match
  \item That can be relied upon and used
  \end{itemize}
  \begin{lstlisting}
   In[]: for each in extensions.keys():
   ....:     print each, "-->", extensions[each]
   ....:
  \end{lstlisting}
\end{frame}

\section{Sets}

\begin{frame}[fragile]
  \frametitle{Creating Sets}
  \begin{lstlisting}
  In[]: a_list = [1, 2, 1, 4, 5, 6, 2]
  In[]: a = set(a_list)
  In[]: a
  \end{lstlisting}
  \begin{itemize}
  \item Conceptually identical to the sets in mathematics
  \item Duplicate elements not allowed
  \item No ordering of elements exists
  \end{itemize}
\end{frame}

\begin{frame}[fragile]
  \frametitle{Operations on Sets}
  \begin{lstlisting}
  In[]: f10 = set([1, 2, 3, 5, 8])
  In[]: p10 = set([2, 3, 5, 7])
  \end{lstlisting}
  \begin{itemize}
  \item Mathematical operations performed on sets, can be performed
  \end{itemize}
  \begin{itemize}
  \item Union
    \begin{lstlisting}
    In[]: f10 | p10
    \end{lstlisting}
  \item Intersection
    \begin{lstlisting}
    In[]: f10 & p10
    \end{lstlisting}
  \item Difference
    \begin{lstlisting}
    In[]: f10 - p10
    \end{lstlisting}
  \item Symmetric Difference
    \begin{lstlisting}
    In[]: f10 ^ p10
    \end{lstlisting}
  \end{itemize}
\end{frame}

\begin{frame}[fragile]
  \frametitle{Sub-sets}
  \begin{itemize}
  \item Proper Subset
    \begin{lstlisting}
    In[]: b = set([1, 2])
    In[]: b < f10
    \end{lstlisting}
  \item Subsets
    \begin{lstlisting}
    In[]: f10 <= f10
    \end{lstlisting}
  \end{itemize}
\end{frame}

\begin{frame}[fragile]
  \frametitle{Elements of sets}
  \begin{itemize}
  \item Containership
    \begin{lstlisting}
    In[]: 1 in f10
    In[]: 4 in f10
    \end{lstlisting}
  \item Iterating over elements
    \begin{lstlisting}
    In[]: for i in f10:
    ....:      print i,
    ....:
    \end{lstlisting}
  \item Subsets
    \begin{lstlisting}
    In[]: f10 <= f10
    \end{lstlisting}
  \end{itemize}
\end{frame}

\begin{frame}[fragile]
  \frametitle{Sets -- Example}
  \begin{block}{}
    Given a list of marks, \texttt{[20, 23, 22, 23, 20, 21, 23]} list
    all the duplicates
  \end{block}
  \begin{lstlisting}
  In[]: marks = [20, 23, 22, 23, 20, 21, 23] 
  In[]: marks_set = set(marks)
  In[]: for mark in marks_set:
  ....:      marks.remove(mark)

    # left with only duplicates
  In[]: duplicates = set(marks)
  \end{lstlisting}
\end{frame}

