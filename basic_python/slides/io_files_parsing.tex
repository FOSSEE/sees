\section{I/O}

\begin{frame}[fragile]
  \frametitle{Printing}
  \begin{lstlisting}
    a = "This is a string"
    a
    print a
  \end{lstlisting}
  \begin{itemize}
  \item Both \texttt{a}, and \texttt{print a} are showing the value
  \item What is the difference?
  \item Typing \texttt{a} shows the value; \texttt{print a} prints it
  \item Typing \texttt{a} shows the value only in interpreter
  \item In a script, it has no effect. 
  \end{itemize}
  \begin{lstlisting}
    b = "A line \n New line"
    b
    print b
  \end{lstlisting}
\end{frame}

\begin{frame}[fragile]
  \frametitle{String formatting}
  \begin{lstlisting}
    x = 1.5
    y = 2
    z = "zed"
    print "x is %2.1f y is %d z is %s" %(x, y, z)
  \end{lstlisting}
\end{frame}

\begin{frame}[fragile]
  \frametitle{\texttt{print x} \& \texttt{print x,}}
  \begin{itemize}
  \item Open an editor
  \item Type the following code
  \item Save as \texttt{print\_example.py}
  \end{itemize}
  \begin{lstlisting}
    print "Hello"
    print "World"

    print "Hello",
    print "World"
  \end{lstlisting}
  \begin{itemize}
  \item Run the script using \texttt{\% run print\_example.py}
  \item \texttt{print x} adds a newline whereas \texttt{print x,} adds
    a space
  \end{itemize}
\end{frame}

\begin{frame}[fragile]
  \frametitle{\texttt{raw\_input}}
  \begin{lstlisting}
    ip = raw_input()    
  \end{lstlisting}
  \begin{itemize}
  \item The cursor is blinking; waiting for input
  \item Type \texttt{an input} and hit <ENTER>
  \end{itemize}
  \begin{lstlisting}
    print ip
  \end{lstlisting}
\end{frame}

\begin{frame}[fragile]
  \frametitle{\texttt{raw\_input} \ldots}
  \begin{lstlisting}
    c = raw_input()
    5.6
    c
    type(c)
  \end{lstlisting}
  \begin{itemize}
  \item \alert{\texttt{raw\_input} always takes a string}
  \end{itemize}
  \begin{lstlisting}
    name = raw_input("Please enter your name: ")
    George
  \end{lstlisting}
  \begin{itemize}
  \item \texttt{raw\_input} can display a prompt string for the user
  \end{itemize}
\end{frame}

\section{Files}

\begin{frame}[fragile]
  \frametitle{Opening files}
  \begin{lstlisting}
    pwd # present working directory
    cd /home/fossee # go to location of the file
  \end{lstlisting}
  {\tiny The file is in our present working directory}
  \begin{lstlisting}
    f = open('pendulum.txt')
    f
  \end{lstlisting}
  \begin{itemize}
  \item \texttt{f} is a file object
  \item Shows the mode in which the file is open (read mode)
  \end{itemize}
\end{frame}

\begin{frame}[fragile]
  \frametitle{Reading the whole file}
  \begin{lstlisting}
    pend = f.read()
    print pend
  \end{lstlisting}
  \begin{itemize}
  \item We have read the whole file into the variable \texttt{pend}
  \end{itemize}
  \begin{lstlisting}
    type(pend)
    pend_list = pend.splitlines()
    pend_list
  \end{lstlisting}
  \begin{itemize}
  \item  \texttt{pend} is a string variable
  \item We can split it at the newline characters into a list of
    strings
  \item Close the file, when done; Also, if you want to read again
  \end{itemize}
  \begin{lstlisting}
    f.close()
    f
  \end{lstlisting}
\end{frame}

\begin{frame}[fragile]
  \frametitle{Reading line-by-line}
  \begin{lstlisting}
    for line in open('pendulum.txt'):
        print line
  \end{lstlisting}
  \begin{itemize}
  \item The file object is an ``iterable''
  \item We iterate over it and print each line
  \item Instead of printing, collect lines in a list
  \end{itemize}
  \begin{lstlisting}
    line_list = [ ]
    for line in open('pendulum.txt'):
        line_list.append(line)
  \end{lstlisting}
\end{frame}


\begin{frame}[fragile]
  \frametitle{File parsing -- Problem}
  \begin{lstlisting}
    A;010002;ANAND R;058;037;42;35;40;212;P;;
  \end{lstlisting}
  \begin{itemize}
  \item File with records like the one above is given
  \item Each record has fields separated by ;
  \item region code; roll number; name; 
  \item marks --- $1^{st}$ L; $2^{nd}$ L; math; science; social; total
  \item pass/fail indicated by P/F; W if withheld and else empty
    \end{itemize}

  \begin{itemize}
  \item We wish to calculate mean of math marks in region B
  \end{itemize}
\end{frame}

\begin{frame}[fragile]
  \frametitle{Tokenization}
  \begin{lstlisting}
    line = "parse this           string"
    line.split()
  \end{lstlisting}
  \begin{itemize}
  \item Original string is split on white-space (if no argument)
  \item Returns a list of strings
  \item It can be given an argument to split on that argrument
  \end{itemize}
  \begin{lstlisting}
    record = "A;015163;JOSEPH RAJ S;083;042;47;AA;72;244;;;"
    record.split(';')
  \end{lstlisting}
\end{frame}

\begin{frame}[fragile]
  \frametitle{Tokenization \ldots}
  \begin{itemize}
  \item Since we split on commas, fields may have extra spaces at ends
  \item We can strip out the spaces at the ends
  \end{itemize}
  \begin{lstlisting}
    word = "     B    "
    word.strip()
  \end{lstlisting}
  \begin{itemize}
  \item \texttt{strip} is returning a new string 
  \end{itemize}
\end{frame}

\begin{frame}[fragile]
  \frametitle{\texttt{str} to \texttt{float}}
  \begin{itemize}
  \item After tokenizing, the marks we have are strings
  \item We need numbers to perform math operations
  \end{itemize}
  \begin{lstlisting}
    mark_str = "1.25"
    mark = int(mark_str)
    type(mark_str)
    type(mark)
  \end{lstlisting}
  \begin{itemize}
  \item \texttt{strip} is returning a new string 
  \end{itemize}
\end{frame}

\begin{frame}[fragile]
  \frametitle{File parsing -- Solution}
  \begin{lstlisting}
    math_B = [] # empty list to store marks
    for line in open("sslc1.txt"):
        fields = line.split(";")

        reg_code = fields[0]
        reg_code_clean = reg_code.strip()

        math_mark_str = fields[5]
        math_mark = float(math_mark_str)

        if reg_code == "B":
            math_B.append(math_mark)

    math_B_mean = sum(math_B) / len(math_B)
    math_B_mean
  \end{lstlisting}
\end{frame}
