\section{The Language}
\begin{frame}[fragile]
  \frametitle{Python!}
  \begin{itemize}
  \item Programming Language
  \item Powerful, High-level, Interpreted, Multi-Platform
  \item Elegant and highly readable syntax
  \item Efficient high-level data structures
  \end{itemize}
  \begin{itemize}
  \item Easy to learn
  \item Allows to concentrate on the problem instead of the language
  \item Increased Productivity
  \end{itemize}
  \begin{itemize}
  \item Guido van Rossum -- BDFL
  \item Conceived in December 1989
  \item Named after ``Monty Python's Flying Circus'', a 70s comedy
  \end{itemize}
\end{frame}

\begin{frame}[fragile]
  \frametitle{Why Python?}
  \begin{itemize}
  \item Extremely readable; Forces programmers to write readable code. 
  \item Interactive; Offers a very fast edit-test-debug cycle.
  \item Doesn't get in your way; High-level data structures let you
    focus on the problem
  \item Handles memory management
  \item Batteries included; Huge standard library for wide range of
    tasks. 
  \item Object-oriented.
  \item C, C++ and FORTRAN interfacing allows use of legacy code
  \item Your time is more valuable than machine time!
  \end{itemize}
\end{frame}

\section{The Interpreter}
\begin{frame}[fragile]
  \frametitle{Python interpreter} 
  \begin{itemize}
  \item Let's get our hands dirty!
  \item Start Python from your shell
  \end{itemize}
  \lstset{language=sh}
  \begin{lstlisting}
    $ python
  \end{lstlisting} %$
  \begin{lstlisting}
Python 2.7.1 (r271:86832, Feb 21 2011, 01:28:26) 
[GCC 4.5.2 20110127 (prerelease)] on linux2
Type "help", "copyright", "credits" or "license" for more information.
>>>
  \end{lstlisting}
  \begin{itemize}
  \item First line shows Python version (2.7.1)
  \item \verb+>>>+ the interpreter's prompt
  \item The interpreter is ready and waiting for your command!
  \end{itemize}
\end{frame}

\begin{frame}[fragile]
  \frametitle{Hello World!}
  \begin{itemize}
  \item Type\texttt{print `Hello World'} and hitting enter
  \end{itemize}
  \begin{lstlisting}
    >>> print 'Hello, World!'
    Hello, World!
  \end{lstlisting}
  \begin{itemize}
  \item The interpreter prints out the words \emph{Hello World}
  \end{itemize}
  \begin{itemize}
  \item Hit \texttt{Ctrl-D} to exit the interpreter
  \item We shall look at IPython, an enhanced interpreter
  \end{itemize}
\end{frame}

\begin{frame}[fragile]
  \frametitle{Versions}
  Before moving on \ldots
  \begin{itemize}
  \item Currently has two stable branches or versions, 2.x and 3.x
  \item 3.x is not backward compatible 
  \item 3.x is deemed to be the future of Python
  \item But, we shall stick to 2.x for this course
  \item The ecosystem around Python 2.x hasn't yet moved to 3.x
  \end{itemize}
\end{frame}

\subsection*{IPython}

\begin{frame}[fragile]
  \frametitle{Invoking IPython}
  \begin{itemize}
  \item An enhanced Python interpreter 
  \item Tab-completion, Easier access to help, Better history
  \end{itemize}
  \lstset{language=sh}
  \begin{lstlisting}
    $ ipython
  \end{lstlisting} %$
  \alert{If \texttt{ipython is not installed}, you need to install it!}
  \begin{itemize}
  \item The prompt is \texttt{In [1]:} instead of \verb+>>>+
  \item \texttt{In} stands for input, 1 indicates the command number
  \item Try \texttt{Hello World}
  \end{itemize}
  \begin{lstlisting}
    In []: print 'Hello, World!'
    Out[]: Hello, World!
  \end{lstlisting}
  {\tiny the numbers have been omitted to avoid confusion}
  \begin{itemize}
  \item Hit \texttt{Ctrl-D} to exit \texttt{ipython}; Say \texttt{y}
    when prompted. 
  \end{itemize}
\end{frame}

\begin{frame}[fragile]
  \frametitle{Getting comfortable}
  \begin{itemize}
  \item Let's try some simple math to get comfortable
  \end{itemize}
  \begin{lstlisting}
    In []: 1 + 2
    In []: 5 - 3
    In []: 7 - 4
    In []: 6 * 5
  \end{lstlisting}
  \begin{itemize}
  \item We get back the expected output
  \item Output is displayed with an \texttt{Out[]}
  \end{itemize}
\end{frame}

\begin{frame}[fragile]
  \frametitle{History \& Arrow Keys}
  \begin{itemize}
  \item Change the \texttt{print 1+2}
  \item Use <UP-Arrow> to go back to \texttt{1+2} command
  \item Use <LEFT-Arrow> to get to start of line; type \texttt{print }
  \item Hit <RETURN>
  \end{itemize}
  \begin{lstlisting}
    In []: print 1 + 2
  \end{lstlisting}
  \begin{itemize}
  \item Now, change the previous command to \texttt{print 10*2}
  \end{itemize}
\end{frame}

\begin{frame}[fragile]
  \frametitle{Tab-Completion}
  \begin{itemize}
  \item We want to use \texttt{round} function
  \item Type \texttt{ro}, and hit <TAB>
  \end{itemize}
  \begin{lstlisting}
    In []: ro<TAB>
  \end{lstlisting}
  \begin{itemize}
  \item Type \texttt{r}, and hit <TAB>
  \item All possibilities are listed out, when ambiguous
  \end{itemize}
\end{frame}

\begin{frame}[fragile]
  \frametitle{\texttt{?} for Help}
  \begin{itemize}
  \item To get help for \texttt{abs} function
  \end{itemize}
  \begin{lstlisting}
    In []: abs?
    In []: abs(19)
    In []: abs(-10.5)
  \end{lstlisting}
  \begin{itemize}
  \item Look at documentation for \texttt{round}
  \item Optional arguments are denoted with square brackets
    \texttt{[]} 
  \end{itemize}
  \begin{lstlisting}
    In []: round(2.484)
    In []: round(2.484, 1)
    In []: round(2.484, 2)
  \end{lstlisting}
\end{frame}

\begin{frame}[fragile]
  \frametitle{\texttt{?} for Help}
  \begin{itemize}
  \item To get help for \texttt{abs} function
  \end{itemize}
  \begin{lstlisting}
    In []: abs?
    In []: abs(19)
    In []: abs(-10.5)
  \end{lstlisting}
  \begin{itemize}
  \item Look at documentation for \texttt{round}
  \item Optional arguments are denoted with square brackets
    \texttt{[]} 
  \end{itemize}
  \begin{lstlisting}
    In []: round(2.484)
    In []: round(2.484, 1)
    In []: round(2.484, 2)
  \end{lstlisting}
\end{frame}

\begin{frame}[fragile]
  \frametitle{Interrupting}
  \begin{lstlisting}
    In []: round(2.484
      ...: 
  \end{lstlisting}
  \begin{itemize}
  \item The \ldots prompt is the continuation prompt
  \item It comes up, since we haven't completed previous command
  \item Either complete by typing the missing \texttt{)}
  \item OR hit \texttt{Ctrl-C} to interrupt the command
  \end{itemize}
  \begin{lstlisting}
    In []: round(2.484
      ...: ^C
  \end{lstlisting}
\end{frame}

\section{Basic Datatypes and Operators}

\begin{frame}[fragile]
  \frametitle{Basic Datatypes}
  \begin{itemize}
  \item Numbers
    \begin{itemize}
    \item int
    \item float
    \item complex
    \end{itemize}
  \item Boolean
  \item Sequence
    \begin{itemize}
    \item Strings
    \item Lists
    \item Tuples
    \end{itemize}
  \end{itemize}
\end{frame}

\begin{frame}[fragile]
  \frametitle{\texttt{int}}
  \begin{lstlisting}
    In []: a = 13
    In []: a
  \end{lstlisting}
  \begin{itemize}
  \item \texttt{a} is a variable of the \texttt{int} type
  \item Use the \texttt{type} command to verify
  \end{itemize}
  \begin{lstlisting}
    In []: type(a)
  \end{lstlisting}
  \begin{itemize}
  \item Integers can be arbitrarily long
  \end{itemize}
  \begin{lstlisting}
    In []: b = 9999999999999999999999999999
    In []: b
  \end{lstlisting}
\end{frame}

\begin{frame}[fragile]
  \frametitle{\texttt{float}}
  \begin{lstlisting}
    In []: p = 3.141592
    In []: p
  \end{lstlisting}
  \begin{itemize}
  \item Decimal numbers are represented using the \texttt{float} type
  \item Notice the loss of precision
  \item Floats have a fixed precision
  \end{itemize}
\end{frame}

\begin{frame}[fragile]
  \frametitle{\texttt{complex}}
  \begin{lstlisting}
    In []: c = 3+4j
  \end{lstlisting}
  \begin{itemize}
  \item A complex number with real part 3, imaginary part 4
  \end{itemize}
  \begin{lstlisting}
    In []: c.real
    In []: c.imag
    In []: abs(c)
  \end{lstlisting}
  \begin{itemize}
  \item It's a combination of two floats
  \item \texttt{abs} gives the absolute value
  \end{itemize}
\end{frame}

\begin{frame}[fragile]
  \frametitle{Operations on numbers}
  \begin{lstlisting}
    In []: 23 + 74
    In []: 23 - 56
    In []: 45 * 76

    In []: 8 / 3 
    In []: 8.0 / 3
    In []: float(8) / 3
  \end{lstlisting}
  \begin{itemize}
  \item The first division is an integer division
  \item To avoid integer division, at least one number should be float
  \item \texttt{float} function is changing int to float
  \end{itemize}
  \begin{lstlisting}
    In []: 87 % 6
    In []: 7 ** 8
  \end{lstlisting} 
  \begin{itemize}
  \item \texttt{\%} is used for modulo operation
  \item \texttt{**} is used for exponentiation
  \end{itemize}
\end{frame}

\begin{frame}[fragile]
  \frametitle{Variables \& assignment}
  \begin{itemize}
  \item All the operations could be done on variables
  \end{itemize}
  \begin{lstlisting}
    In []: a = 23 
    In []: b = 74
    In []: a * b  
                  
    In []: c = 8
    In []: d = 8.0
    In []: f = c / 3
  \end{lstlisting}
  \begin{itemize}
  \item Last two commands show assignment
  \end{itemize}
  \begin{lstlisting}
    In []: c = c / 3
  \end{lstlisting}
  An operation like the one above, may equivalently be written as
  \begin{lstlisting}
    In []: c /= 3
  \end{lstlisting}
\end{frame}

\begin{frame}[fragile]
  \frametitle{Booleans \& Operations}
  \begin{itemize}
  \item All the operations could be done on variables
  \end{itemize}
  \begin{lstlisting}
    In []: t = True
    In []: t
    In []: f = not t
    In []: f
    In []: f or t
    In []: f and t
  \end{lstlisting}
  \begin{itemize}
  \item Multiple operation in a single command
  \item We use parenthesis for explicitly stating what we mean
  \item No discussion of operator precedence
  \end{itemize}
  \begin{lstlisting}
    In []: (f and t) or t
    In []: f and (t or t)
  \end{lstlisting}
\end{frame}

\begin{frame}[fragile]
  \frametitle{Sequences}
  \begin{itemize}
  \item Hold a bunch of elements in a sequence
  \item Elements are accessed based on position in the sequence
  \item The sequence data-types
    \begin{itemize}
    \item str
    \item list
    \item tuple
    \end{itemize}
  \end{itemize}
\end{frame}

\begin{frame}[fragile]
  \frametitle{Strings, Lists \& Tuples}
  \begin{itemize}
  \item Anything withing quotes is a string
  \end{itemize}
  \begin{lstlisting}
    In []: greet_str = "hello"
  \end{lstlisting}
  \begin{itemize}
  \item Items enclosed in \texttt{[ ]} and separated by \texttt{,}s
    constitute a list
  \end{itemize}
  \begin{lstlisting}
    In []: num_list = [1, 2, 3, 4, 5, 6, 7, 8]
  \end{lstlisting}
  \begin{itemize}
  \item Items of a tuple are enclosed by \texttt{( )} instead of
    \texttt{[ ]}  
  \end{itemize}
  \begin{lstlisting}
    In []: num_tuple = (1, 2, 3, 4, 5, 6, 7, 8)
  \end{lstlisting}
\end{frame}

\begin{frame}[fragile]
  \frametitle{Operations on Sequences}
  \begin{itemize}
  \item Accessing elements
  \end{itemize}
  \begin{lstlisting}
    In []: num_list[2] 
    In []: num_tuple[2]
    In []: greet_str[2]
  \end{lstlisting}
  \begin{itemize}
  \item Add two sequences of same type
  \end{itemize}
  \begin{lstlisting}
    In []: num_list + [3, 4, 5, 6]
    In []: greet_str + " world!"  
  \end{lstlisting}
  \begin{itemize}
  \item Get the length of a sequence
  \end{itemize}
  \begin{lstlisting}
    In []: len(num_list) 
    In []: len(greet_str)
  \end{lstlisting}
\end{frame}

\begin{frame}[fragile]
  \frametitle{Operations on Sequences \ldots}
  \begin{itemize}
  \item Check for container-ship of  elements
  \end{itemize}
  \begin{lstlisting}
    In []: 3 in num_list
    In []: 'h' in greet_str
    In []: 'w' in greet_str
    In []: 2 in num_tuple
  \end{lstlisting}
  \begin{itemize}
  \item Finding maximum and minimum 
  \end{itemize}
  \begin{lstlisting}
    In []: max(num_list)
    In []: min(greet_str)
  \end{lstlisting}
  \begin{itemize}
  \item Slice a sequence
  \end{itemize}
  \begin{lstlisting}
    In []: num_list[1:5]
  \end{lstlisting}
  \begin{itemize}
  \item Stride over a sequence
  \end{itemize}
  \begin{lstlisting}
    In []: num_list[1:8:2]
  \end{lstlisting}
\end{frame}

